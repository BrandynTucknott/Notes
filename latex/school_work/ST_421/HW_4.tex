\documentclass{exam}

\usepackage{amsmath,amssymb,amsfonts,amsthm,dsfont}
\usepackage{lib/extra}
\usepackage{graphicx}
\usepackage{tikz}
\usepackage{enumitem}
\usepackage{pgfplots}

\title{ST 421 HW 4}
\author{Brandyn Tucknott}
\date{1 November 2024}

\begin{document}
\maketitle

\begin{questions}
    \textbf{3.96 }
The telephone lines serving an airline reservation office are all busy around 60\% of the time.

\newline
\textbf{(a) } If you are calling this office, what is the probability that you will complete your call on the first try? Second try? Third try?
\sol
Note that the probability of completing the call is $p = 1 - q = 0.4$ with $q = 0.6$.
$$P(\text{calls } = 1) = p \cdot q^0 = 0.4$$
$$P(\text{calls } = 2) = p \cdot q^1 = 0.24$$
$$P(\text{calls } = 3) = p \cdot q^2 = 0.144$$

\newline
\textbf{(b) } If you and a friend must both complete calls to this office, what is the probability that a total of four tries will be necessary for you both to get through?
\sol
To find this, we use the Negative Binomial Distribution.
$$P(X = 4) = \binom{k - 1}{r - 1}p^rq^{k-r} = \binom{3}{1}p^2q^2 = 3 \cdot 0.4^2 \cdot 0.6^2 = 0.1728$$

\newpage
\textbf{3.105 }
In southern California, a growing number of individuals pursuing teaching credentials are choosing paid internships over traditional student teaching programs. A group of eight candidates for three local teaching positions consisted of five who had enrolled in paid internships and three who enrolled in traditional student teaching programs. All eight candidates appear to be equally qualified, so three are randomly selected to fill the open positions. Let $Y$ be the number of internship trained candidates who are hired.

\newline

\textbf{(a) } Does $Y$ have a binomial or hypergeometric distribution? Why?
\sol
$Y$ has a hypergeometric distribution because you are sampling from a finite pool without replacement.

\newline
\textbf{(b) } Find the probability two or more internship trained candidates are hired.
\sol
$P(Y \geq 2) = P(Y = 2) + P(Y = 3) = \frac{\binom{5}{2}\binom{3}{1} + \binom{5}{3}\binom{3}{0}}{\binom{8}{3}} \approx 0.714$

\textbf{(c) } What are the mean and standard deviation of $Y$?
\sol
For a hypergeometric distribution $(N, M, K)$, we know the mean and variance are 
$$\mu = \frac{MK}{N}, \sigma^2 = \mu\frac{(N - M)(N - K)}{N(N - 1)}$$

For our particular problem, we have $\text{Hypergeometric}(N = 8, M = 5, K = 3)$ which implies that our mean and variance are
$$\mu = \frac{KM}{N} = \frac{3 \cdot 5}{8} = \frac{15}{8} = 1.875$$
$$\sigma = \sqrt{\mu\frac{(n - M)(N - K)}{N(N - 1)}} = \sqrt{\frac{15}{8} \cdot \frac{3 \cdot 5}{8 \cdot 7}} \approx 0.709$$

\newpage
\textbf{3.121 }
Let $Y$ denote a random variable that has a Poisson distribution with mean $\lambda = 2$. Find the following:

\newline
\textbf{(a) } $P(Y = 4)$.
\sol
$$P(Y = 4) = e^{-\lambda} \cdot \frac{\lambda^y}{y!} = e^{-2} \cdot \frac{2^4}{4!} \approx 0.090$$

\newline
\textbf{(b) } $P(Y \geq 4)$.
\sol
$$P(Y \geq 4) = 1 - P(Y \leq 3) = 1 - P(Y = 0) - P(Y = 1) - P(Y = 2) - P(Y = 3) =$$
$$1 - e^{-\lambda} \paren{\frac{\lambda^0}{0!} + \frac{\lambda^1}{1!} + \frac{\lambda^2}{2!} + \frac{\lambda^3}{3!}} = 1 - e^{-2} \paren{1 + \frac{2}{1} + \frac{2^2}{2} + \frac{2^3}{6}} \approx 0.143$$

\newline
\textbf{(c) } $P(Y < 4)$.
\sol
$$P(Y < 4) = P(Y \leq 3) = e^{-\lambda} \paren{\frac{\lambda^0}{0!} + \frac{\lambda^1}{1!} + \frac{\lambda^2}{2!} + \frac{\lambda^3}{3!}} = e^{-2} \paren{1 + \frac{2}{1} + \frac{2^2}{2} + \frac{2^3}{6}} \approx 0.857$$

\newline
\textbf{(d) } $P(Y \geq 4 | Y \geq 2)$.
\sol
$$P(Y \geq 4 | Y \geq 2) = \frac{P(Y \geq 4 \cap Y \geq 2)}{P(Y \geq 2)} = \frac{P(Y \geq 4)}{P(Y \geq 2)} =$$
$$\frac{0.143}{1 - e^{-2}\paren{\frac{2^0}{0!} + \frac{2^1}{1!}}} \approx \frac{0.143}{0.594} \approx 0.241$$

\newpage
\textbf{3.139 }
In the daily production of a certain kind of rope, the number of defects per foot $Y$ is assumed to have a poisson distribution with mean $\lambda = 2$. The profit per foot when the rope is sold is given by $X = 50 - 2Y - Y^2$. Find the expected profit per foot.

\sol
$$E(X) = E(50 - 2Y - Y^2) = E(50) - E(2Y) - E(Y^2) = 50 - 2E(Y) - E(Y^2)$$
Recall that 
$$\text{Var}(Y) = E(Y^2) - E^2(Y) = E(Y^2) - \lambda^2 \longrightarrow$$
$$\lambda = E(Y^2) - \lambda^2 \longrightarrow$$
$$E(Y^2) = \lambda^2 + \lambda$$
Plugging this into our original equation, we are left with
$$E(X) = 50 - 2E(Y) - E(Y^2) = 50 - 2\lambda - (\lambda^2 + \lambda) = 50 - 2\lambda - \lambda^2 - \lambda = 50 - 3\lambda - \lambda^2$$

Subbing in $\lambda = 2$, we have that
$$E(X) = 50 - 3(2) - (2)^2 = 40$$

\newpage
\textbf{3.147 }
If $Y$ has a geometric distribution with a probability of success $p$, show that the moment-generating function for $Y$ is
$$m(t) = \frac{pe^t}{1 - qe^t}, \text{where }q = 1 - p$$
\sol
A moment generating is function is defined as $E(e^{tX})$ such that the $n^{th}$ moment is defined by 
\newline
$E(X^n) = \frac{d^n}{dt^n} E\paren{e^{tX}}\Big|_{t = 0}$. So we start by expanding this out.
$$E(e^{tX}) = \sum_{k = 1}^\infty e^{tk} \cdot pq^{k - 1} = \frac{p}{q}\sum_{k = 1}^\infty e^{tk}q^k = \frac{p}{q}\sum_{k = 1}^\infty (e^tq)^k = \frac{p}{q}\sum_{k = 0}^\infty (e^tq)^k - \frac{p}{q}\paren{e^tq}^0 =$$
$$= \frac{p}{q}\sum_{k = 0}^\infty (e^tq)^k - \frac{p}{q} = \frac{p}{q}\frac{1}{1 - qe^t} - \frac{p}{q} = \frac{p - p\paren{1 - qe^t}}{q\paren{1 - qe^t}} = \frac{p - p + pqe^t}{q\paren{1 - qe^t}} = \frac{pe^t}{1 - qe^t}$$

Note that we assume $\abs{qe^t} < 1$ so that we can calculate a finite value for a converging series. If $E\paren{e^{tX}}$ does not converge, then we are not necessarily guaranteed meaningful and well-defined moments. This is why we assume the series converges.

\newpage

\textbf{3.153 }
Find the distributions of the random variables that each have the following moment generating functions:

\newline
\textbf{(a) }$m(t) = \paren{\frac{1}{3}e^t + \frac{2}{3}}^5$
\sol
Binomial Distribution.

\newline
\textbf{(b) }$m(t) = \frac{e^t}{2 - e^t}$
\sol
Geometric Distribution.

\newline
\textbf{(c) }$m(t) = e^{2\paren{e^t - 1}}$
\sol
Poisson Distribution.


\newpage
\textbf{4.1 }
Let $Y$ be a random variable with $P(y)$ given in the table below.

\newline
\begin{tabular}{c|cccc}
    $y$ & 1 & 2 & 3 & 4 \\
    \hline
    $P(y) $ & 0.4 & 0.3 & 0.2 & 0.1 \\
\end{tabular}
\newline

\textbf{(a) } Give the distribution function $F(y)$. Be sure to specify the value of $F(y)$ for all $y \in (-\infty, \infty)$.
\sol
\begin{tabular}{c|ccccc}
    $y \in $[a, b)$ & [0, 1) & [1, 2) & [2, 3) & [3, 4) & [4, \infty) \\
    \hline
    $F(y)$ & 0 & 0.4 & 0.7 & 0.9 & 1 \\
\end{tabular}
\newline
\newline


\textbf{(b) } Sketch the distribution function given in Part (a).
\sol
\newline
\begin{tikzpicture}
    \begin{axis}[
        axis x line=middle,
        axis y line=middle,
        xmin=0, xmax=5, % Adjust x-axis range as needed
        ymin=0, ymax=1.2, % Adjust y-axis range as needed
        xtick={1, 2, 3, 4},
        ytick={0, 0.4, 0.7, 0.9, 1},
        xlabel={$y$},
        ylabel={$F(y)$},
        yticklabels={0, 0.4, 0.7, 0.9, 1},
        samples=100,
        domain=0:5
    ]
    % Plot each piece of the function with horizontal line segments and step jumps
    \addplot[blue, thick, domain=0:1, samples=2] {0}; % F(y) = 0 for y < 1
    \addplot[blue, thick, domain=1:2, samples=2] {0.4}; % F(y) = 0.4 for 1 <= y < 2
    \addplot[blue, thick, domain=2:3, samples=2] {0.7}; % F(y) = 0.7 for 2 <= y < 3
    \addplot[blue, thick, domain=3:4, samples=2] {0.9}; % F(y) = 0.9 for 3 <= y < 4
    \addplot[blue, thick, domain=4:5, samples=2] {1}; % F(y) = 1 for y >= 4

    % Add open circles for the jumps
    \addplot[blue, only marks, mark=o] coordinates {(1,0) (2,0.4) (3,0.7) (4,0.9)};

    % Add closed circles at the endpoints
    \addplot[blue, only marks, mark=*] coordinates {(1,0.4) (2,0.7) (3,0.9) (4,1)};
    \end{axis}
\end{tikzpicture}

\newpage
\textbf{4.8 }
Suppose $Y$ has a density function
$$f(y) =\begin{cases}
    ky(1 - y) & 0 \leq y \leq 1 \\
    0 & \text{elsewhere} \\
\end{cases}$$

\newline
\textbf{(a) } Find the value $k$ that makes $f(y)$ a probability density function.
\sol
$$F_Y(y \rightarrow \infty) = \int_{-\infty}^\infty kt(1 - t)dt = k\int_{0}^1 t - t^2dt = k\paren{\frac{t^2}{2} - \frac{t^3}{3}}\Bigg|_0^1 = k\paren{\frac{1}{2} - \frac{1}{3}} = \frac{k}{6} = 1$$
From this we conclude that $k = 6$.

Additionally, for future parts we will calculate $F_Y(y)$ here.
$$F_Y(y) = \int_{-\infty}^y kt(1 - t)dt = k\int_0^yt - t^2dt = k\paren{\frac{t^2}{2} - \frac{t^3}{3}}\Bigg|_0^y = k\paren{\frac{y^2}{2} - \frac{y^3}{3}}$$

\newline
\textbf{(b) } Find $P(0.4 \leq Y \leq 1)$.
\sol
$$P(0.4 \leq Y \leq 1) = F_Y(1) - F_Y(0.4) = 6\paren{\frac{1}{2} - \frac{1}{3}} - 6\paren{\frac{0.4^2}{2} - \frac{0.4^3}{3}} = 0.648$$

\newline
\textbf{(c) } Find $P(0.4 \leq Y < 1)$.
\sol
Since $Y$ is a continuous random variable, the process is exactly the same as in Part (b), giving us
$$P(0.4 \leq Y < 1) = 0.648$$

\newline
\textbf{(d) } Find $P(Y \leq 0.4 | Y \leq 0.8)$.
\sol
By Bayes theorem, we evaluate the probability to be
$$P(Y \leq 0.4 | Y \leq 0.8) = \frac{P(Y \leq 0.4 \cap Y \leq 0.8)}{P(Y \leq 0.8)} = \frac{P(Y \leq 0.4)}{P(Y \leq 0.8)} =$$
$$= \frac{F_Y(0.4)}{F_Y(0.8)} = \frac{6\paren{\frac{0.4^2}{2} - \frac{0.4^3}{3}}}{6\paren{\frac{0.8^2}{2} - \frac{0.8^3}{3}}} \approx 0.393$$

\newline
\textbf{(e) } Find $P(Y \leq 0.4 | Y < 0.8)$.
\sol
Since $Y$ is a continuous random variable, the process is exactly the same as in Part (d), giving us
$$P(Y \leq 0.4 | Y < 0.8) \approx 0.393$$
\end{questions}

\end{document}