\documentclass{exam}

\usepackage{amsmath,amssymb,amsfonts,amsthm,dsfont}
\usepackage{lib/extra}
\usepackage{graphicx}
\usepackage{tikz}

\title{MTH 463 HW 1}
\author{Brandyn Tucknott}
\date{30 September 2024}

\begin{document}
\maketitle

\begin{questions}
    \question
A deck of only six cards $\cbrac{1, 2, 3, 4, 5, 6}$ is shuffled so that each of the 6! orderings has equal probability of $\frac{1}{6!}$. Let $A$ be the event that the card $1$ is among the top three cards in the deck, and let $B$ be the event that card $5$ is the second card from the top.
\begin{parts}
    \part
    Find $P(A)$ and $P(B)$.
    \\ \textit{Solution.} \\
    $$P(A) = 3\cdot\frac{5!}{6!} = \frac{1}{2}$$
    $$P(B) = \frac{1}{6}$$
    
    \part
    Find $P(A \cap B)$.
    \\ \textit{Solution.} \\
    $$P(A \cap B) = \frac{2\cdot 4!}{6!} = \frac{1}{15}$$
    
    \part
    Find $P(A \cup B)$.
    \\ \textit{Solution.} \\
    $$P(A \cup B) = P(A) + P(B) - P(A \cap B) = \frac{1}{2} + \frac{1}{6} - \frac{1}{15} = \frac{3}{5}$$
    
\end{parts}

\newpage
\question
Consider events $E$, $F$, $G$. Find the expressions in terms of these sets of the following events. In your answers, use $E^c$, $F^c$, $G^c$ if needed to denote the corresponding complementary events.
\begin{parts}
    \part
    At least one of the events $E$, $F$ $G$ occurs.
    \\ \textit{Solution.} \\
    $P(E \cup F \cup G)$
    
    \part
    Exactly one of $E$, $F$, $G$ occurs.
    \\ \textit{Solution.} \\
    $P(E) + P(F) + P(G) - 2P(E \cap F) - 2P(E \cap G) - 2P(F \cap G) + 3P(E \cap F \cap G)$
    
    \part
    At most one of the events $E$, $F$, $G$ occurs.
    \\ \textit{Solution.} \\
    At most one is equivalent to saying the opposite of at least 2. \\
    $1 - \paren{P(E \cap F) + P(E \cap G) + P(F \cap G) - 2P(E \cap F \cap G)}$
    
    \part
    None of $E$, $F$, $G$ occur.
    \\ \textit{Solution.} \\
    $1 - P(E \cup F \cup G)$
    
    \part
    Exactly two of $E$, $F$, $G$ occur.
    \\ \textit{Solution.} \\
    $P(E \cap F) + P(E \cap G) + P(F \cap G) - 3P(E \cap F \cap G)$
    
\end{parts}

\newpage
\question
Show that the probability that \textit{exactly} one of two events $A$ or $B$ occurs is
$$P(A) + P(B) - 2P(A \cap B)$$
\begin{proof}
We know that the probability of $A$ and/or $B$ occurring is
$$P(A \cup B) = P(A) + P(B) - P(A \cap B).$$
If we can remove the case where $A$ and $B$ occur, we will be left with the probability of $A$ xor $B$ occurring. So
$$P(A \oplus B) = P(A \cup B) - P(A \cap B) = P(A) + P(B) - 2P(A \cap B).$$
\end{proof}

\newpage
\question
A fair die has three of its faces painted blue, two painted red, and the remaining painted green. The die is rolled 7 times. Let $X$ denote the number of times that a blue face is rolled. Similarly let $Y$ denote the number of times a  red face is rolled, and $G$ a green face. Find $P(X = 3, Y = 2, G = 2)$.

\\ \textit{Solution.} \\
$$P(X = 3, Y = 2, G = 2) = \binom{7}{3, 2, 2}\paren{\frac{1}{2}}^3\paren{\frac{1}{3}}^2\paren{\frac{1}{6}}^2 = \frac{7!}{3!\cdot2!\cdot2!} \cdot \frac{1}{8} \cdot \frac{1}{9} \cdot \frac{1}{36} = \frac{210}{2592} \approx 0.081$$

\newpage
\question
Consider the polynomial $P(x_1, ..., x_n) = \paren{x_1 + ... + x_n}^r$ and let $Q$ denote the $n^{th}$-order partial derivative of $P$ with respect to $x_1, ..., x_n$, that is,
$$Q = \frac{\partial^n P}{\partial x_1 ... \partial x_n}$$
For $r > n$ find the number of different monomials in $Q$.
\\
\textit{Solution.} \\
This problem is equivalent to creating as many distinct sets of length $r$ using $x_1, ... x_n$. This can be translated into a "stars and bars" problem, which has the distinct occurrences $O$ of
\begin{equation}
    O = \binom{n + r - 1}{r}.
\end{equation}
 Next, we can evaluate $Q$ to be
$$Q = \frac{\partial^n P}{\partial x_1 ... \partial x_n} =$$
$$\frac{\partial^n}{\partial x_1 ... \partial x_n} \paren{x_1 + ... + x_n}^r=$$
$$r \cdot (r - 1) \cdot ... \cdot (r - n + 1) \paren{x_1 + ... + x_n}^{r - n} =$$
$$\frac{r!}{(r - n)!} \paren{x_1 + ... + x_n}^{r - n}.$$

Our goal now is to see how many distinct monomials result from $Q$ being expanded out, which we can count by (1) to be
$$\binom{n + (r - n) - 1}{r - n} = $$
$$\binom{r - 1}{r - n}.$$
\newpage
\end{questions}

\end{document}