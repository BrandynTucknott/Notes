\documentclass{exam}

\usepackage{amsmath,amssymb,amsfonts,amsthm,dsfont}
\usepackage{lib/extra}
\usepackage{graphicx}
\usepackage{tikz}
\usepackage{enumitem}

\title{MTH 463 HW 5}
\author{Brandyn Tucknott}
\date{4 November 2024}

\begin{document}
\maketitle

\begin{questions}
    \question
Assume $X$ is a discrete random variable with the probability mass function

$$m(x) =
\begin{cases}
    \frac{1}{2} & x = 0 \\
    \frac{1}{3} & x = 1 \\
    \frac{1}{6} & x = 2 \\
\end{cases}$$

Find $E(X), E(X^2), \text{and Var}(X)$.

\sol
We calculate the following to be
$$E(X) = \sum_{j = 1}^3 x_j \cdot P(x_j) = 0 \cdot \frac{1}{2} + 1 \cdot \frac{1}{3} + 2 \cdot \frac{1}{6} = \frac{2}{3}$$

$$E(X^2) = \sum_{j = 1}^3 x_j \cdot P(x_j) = 0^2 \cdot \frac{1}{2} + 1^2 \cdot \frac{1}{3} + 2^2 \cdot \frac{1}{6} = 1$$

$$\text{Var}(X) = E(X^2) - E^2(X) = 1 - \paren{\frac{2}{3}}^2 = 1 - \frac{4}{9} = \frac{5}{9}$$


\newpage
\question
Let $X$ be a Binomial random variable with parameters $n, p$. Show that
$$E\paren{\frac{1}{1 + X}} = \frac{1 - (1 - p)^{n + 1}}{(n + 1)p}$$

\begin{proof}
    First, note that since
    $$k \binom{n}{k} = n \binom{n - 1}{k - 1} \longrightarrow$$
    \begin{equation}
        k\binom{n + 1}{k} = (n + 1)\binom{n}{k - 1} \rightarrow
        \binom{n}{k - 1} = \frac{k}{n + 1} \binom{n + 1}{k}
    \end{equation}

    Also recognize that since
    $$1 = 1^{n + 1} = \sum_{k = 0}^{n + 1} \binom{n + 1}{k}p^k(1 - p)^{(n + 1) - k} \longrightarrow$$
    $$1 = \binom{n + 1}{0}p^0(1 - p)^{(n + 1) - 0} + \sum_{k = 1}^{n + 1} \binom{n + 1}{k}p^k(1 - p)^{(n + 1) - k}  = (1 - p)^{n + 1} + \sum_{k = 1}^{n + 1} \binom{n + 1}{k}p^k(1 - p)^{(n + 1) - k}\longrightarrow$$
    \begin{equation}
        1 - (1 - p)^{n + 1} = \sum_{k = 1}^{n + 1} \binom{n + 1}{k}p^k(1 - p)^{(n + 1) - k}
    \end{equation}
    
    
    These are important, and will be used later in the proof. To find $E\paren{\frac{1}{1 + X}}$, we simply compute it. Since $X \sim \text{binomial}(n, p)$,

    $$E\paren{\frac{1}{1 + X}} = \sum_{k = 0}^n \frac{1}{1 + k}\binom{n}{k}p^k(1 - p)^{n - k}) =$$
    $$= \sum_{k = 1}^{n + 1} \frac{1}{1 + (k - 1)}\binom{n}{k - 1}p^{k - 1}(1 - p)^{n - (k - 1)} =$$
    \begin{equation}
        \sum_{k = 1}^{n + 1} \frac{1}{k}\binom{n}{k - 1}p^{k - 1}(1 - p)^{(n + 1) - k}
    \end{equation}

    By substituting equation (1) into equation (3), we get
    $$E\paren{\frac{1}{1 + X}} = \sum_{k = 1}^{n + 1} \frac{1}{k}\paren{\frac{k}{n + 1}\binom{n + 1}{k}}p^{k - 1}(1 - p)^{(n + 1) - k} =$$
    $$= \sum_{k = 1}^{n + 1}\frac{1}{n + 1}\binom{n + 1}{k}p^{k - 1}(1 - p)^{(n + 1) - k} =$$

    \begin{equation}
        \frac{1}{p(n + 1)}\sum_{k = 1}^{n + 1}\binom{n + 1}{k}p^k(1 - p)^{(n + 1) - k}
    \end{equation}

    Another substitution of (2) into (4) yields
    $$E\paren{\frac{1}{1 - X}} = \frac{1}{p(n + 1)} \cdot \paren{1 - (1 - p)^{n + 1}} = \frac{1 - (1 - p)^{n + 1}}{p(n + 1)}$$

    This matches the claim, so we are done.
\end{proof}

\newpage
\question
Let $X$ be a Poisson random variable with parameter $\lambda$, that is for $k = 0, 1, 2, \hdots$
$$P(X = k) = e^{-\lambda} \frac{\lambda^k}{k!}$$

Find $E\paren{\frac{1}{1 + X}}$.

\sol
Recall that
\begin{equation}
    e^x = \sum_{k = 0}^\infty \frac{x^k}{k!}
\end{equation}

Similarly to Question (2), we directly calculate the expected value.
$$E\paren{\frac{1}{1 + X}} = \sum_{k = 0}^\infty \frac{1}{1 + k} e^{-\lambda} \frac{\lambda^k}{k!} = e^{-\lambda} \sum_{k = 0}^\infty \frac{1}{k + 1}\frac{\lambda^k}{k!} \cdot \frac{\lambda}{\lambda} = \frac{e^{-\lambda}}{\lambda}\sum_{k = 0}^\infty \frac{\lambda^{k + 1}}{(k + 1)!} =$$
$$= \frac{e^{-\lambda}}{\lambda}\sum_{k = 0}^\infty \frac{\lambda^k}{k!} - \frac{e^{-\lambda}}{\lambda}\frac{\lambda^0}{0!} = \frac{e^{-\lambda}}{\lambda}\sum_{k = 0}^\infty \frac{\lambda^k}{k!} - \frac{e^{-\lambda}}{\lambda} =$$

\begin{equation}
    = \frac{e^{-\lambda}}{\lambda}\paren{\sum_{k = 0}^\infty \frac{\lambda^k}{k!} - 1}
\end{equation}

We can substitute in equation (5) into equation (6) using $e^\lambda$ instead of $e^x$. This gives us

$$E\paren{\frac{1}{1 + X}} = \frac{e^{-\lambda}}{\lambda}\paren{e^\lambda - 1} = \frac{1 - e^{-\lambda}}{\lambda}$$



\newpage
\question
Let $X$ be a continuous random variable with probability density function given by
$$f_X(x) = c(1 - x^2)\mathbb{1}_{[-1, 1]}(x) =
\begin{cases}
    c(1 - x^2) & \text{for } -1 \leq x \leq 1 \\
    0 & \text{otherwise}
\end{cases}$$

\begin{parts}
    \part % part a
    What is the value of $c$?
    \sol
    $$\int_{-\infty}^\infty f_X(x)dx = \int_{-1}^1 c(1 - x^2)dx = \int_{-1}^1 c dx - \int_{-1}^1 cx^2 dx = 2c - \frac{c}{3} \cdot 2 = \frac{4}{3}c = 1$$

    Solving for $c$ gives $c = \frac{3}{4}$.

    
    \part % part b
    Find the cumulative distribution of $X$.
    \sol
    $$F_X(x) = \int_{-\infty}^x c(1 - t^2) dt = c\paren{t - \frac{t^3}{3}}\Bigg|_{-1}^x = c\paren{\paren{x - \frac{x^3}{3}} - \paren{-1 + \frac{1}{3}}} = \frac{3}{4}\paren{x - \frac{x^3}{3} + \frac{2}{3}}$$
    
    \part % part c
    Find $E(X)$ and $\text{Var}(X)$.
    \sol
    $$E(X) = \int_{-\infty}^\infty x \cdot P(x) dx = \int_{-1}^1 x \cdot c(1 - x^2) dx = c\int_{-1}^1 x - x^3 dx = c \cdot 0 = 0 \text{ since } x - x^3 \text{ is an odd function}$$

    $$\text{Var}(X) = c\int_{-\infty}^\infty x^2 (1 - x^2) dx - E^2(X) = c\int_{-1}^1 x^2 - x^4 dx - 0 = c \cdot 2\paren{\frac{x^3}{3} - \frac{x^5}{5}}\Bigg|_0^1 =$$
    $$= 2c \cdot \frac{2}{15} = \frac{4}{15} \cdot \frac{3}{4} = \frac{3}{15}$$
    
    \part % part d
    Find $P(\abs{X} < \frac{1}{2})$.
    \sol
    $$P\paren{\abs{X} < \frac{1}{2}} = P\paren{-\frac{1}{2} < X < \frac{1}{2}} = P\paren{X < \frac{1}{2}} - P\paren{X < -\frac{1}{2}} = F_X\paren{\frac{1}{2}} - F_X\paren{-\frac{1}{2}} =$$
    $$= \frac{3}{4}\paren{\frac{1}{2} - \frac{1}{3} \cdot \paren{\frac{1}{2}}^3 + \frac{2}{3}} - \frac{3}{4}\paren{-\frac{1}{2} + \frac{1}{3} \cdot \paren{\frac{1}{2}}^3 + \frac{2}{3}} = \frac{3}{4}\paren{\frac{1}{2} - \frac{1}{24} + \frac{2}{3} + \frac{1}{2} - \frac{1}{24} - \frac{2}{3}} = \frac{3}{4} \cdot \frac{11}{12} = \frac{11}{16}$$
    
\end{parts}

\newpage
\question
A real valued random variable $X$ is said to have a Standard Cauchy Distribution if it has pdf
$$f_X(x) = \frac{1}{\pi}\cdot \frac{1}{1 + x^2}$$

\begin{parts}
    \part
    Find $F_X(x)$, the cumulative distribution of $X$. Check that your answer satisfies $\lim_{x \rightarrow -\infty} F_X(x) = 0$ and $\lim_{x \rightarrow \infty} F_X(x) = 1$.
    \sol
    $$F_X(x) = \int_{-\infty}^x f_X(t) dt = \int_{-\infty}^x \frac{1}{\pi} \cdot \frac{1}{1 + t^2} dt = \frac{1}{\pi} \cdot \arctan t \Big|_{-\infty}^x =$$
    $$= \frac{1}{\pi}\paren{\arctan (x) - \arctan (-\infty)} = \frac{\arctan (x)}{\pi} - (-\frac{\pi}{2\pi}) = \frac{\arctan x}{\pi} + \frac{1}{2}$$

    \newline

    We now do a sanity check and see if our answer satisfies the limits as $x$ approaches positive and negative infinity.
    $$\lim_{x \rightarrow -\infty} F_X(x) = \lim_{x \rightarrow -\infty} \frac{\arctan x}{\pi} + \frac{1}{2} = \frac{\arctan(-\infty)}{\pi} + \frac{1}{2} = -\frac{\pi}{2}\frac{1}{\pi} + \frac{1}{2} = 0$$
    $$\lim_{x \rightarrow \infty} F_X(x) = \lim_{x \rightarrow \infty} \frac{\arctan x}{\pi} + \frac{1}{2} = \frac{\arctan(\infty)}{\pi} + \frac{1}{2} = \frac{\pi}{2}\frac{1}{\pi} + \frac{1}{2} = 1$$



    \part
    Show that $Y = \frac{1}{X}$ is also a Standard Cauchy random variable.
    \begin{proof}
        Since $Y = \frac{1}{X}, y = \frac{1}{x} \rightarrow x = \frac{1}{y}$. We wish to show that the pdf for $Y$ is
        $$f_Y(y) = \frac{1}{\pi} \cdot \frac{1}{1 + y^2}$$

        We can accomplish this by writing the cdf for $Y$ in terms of $X$ and taking the derivative with respect to $y$.
        $$F_Y(y) = P(Y \leq y) = P(\frac{1}{X} \leq \frac{1}{x}) = P(X \geq x) = 1 - P(X \leq x) = 1 - F_X(x)$$
        Note that since we are dealing with continuous random variables, $P(X > x)$ and $P(X \geq x)$ are indistinguishable since $\R$ is uncountably infinite $\rightarrow P(X = x) = 0$. Then we compute 
        $$f_Y(y) = \frac{d}{dy}F_Y(y) = \frac{d}{dy}\paren{1 - F_X(x)} = \frac{d}{dy}\paren{1 - \paren{\frac{\arctan (x)}{\pi} + \frac{1}{2}}} =$$
        $$\frac{d}{dy} \paren{1 - \frac{\arctan x}{\pi} - \frac{1}{2}} = \frac{d}{dy} \paren{\frac{1}{2} - \frac{\arctan x}{\pi}} = -\frac{d}{dy} \paren{\frac{\arctan x}{\pi}}$$

        Substituting in $x = \frac{1}{y}$, we end up with 
        $$f_Y(y) = -\frac{d}{dy} \paren{\frac{\arctan \paren{\frac{1}{y}}}{\pi}} = \frac{-1}{\pi} \cdot \paren{\frac{1}{1 + \paren{\frac{1}{y}}^2} \cdot \frac{-1}{y^2}} = \frac{1}{\pi} \cdot \frac{1}{1 + y^2}$$

        This is exactly what we were hoping to show, so we conclude that if $X$ is a Standard Cauchy random variable, then $Y = \frac{1}{X}$ is as well.
    \end{proof}
\end{parts}
\end{questions}

\end{document}