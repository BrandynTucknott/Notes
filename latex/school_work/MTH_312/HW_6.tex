\documentclass{exam}

\usepackage{amsmath,amssymb,amsfonts,amsthm,dsfont}
\usepackage{lib/extra}
\usepackage{graphicx}
\usepackage{tikz}
\usepackage{enumitem}
\usepackage{bbm}
\usepackage{pgfplots}
\usepackage{fontenc}
\usepackage{float}

\pgfplotsset{compat=1.18}

\title{MTH 312 HW 6}
\author{Brandyn Tucknott}
\date{25 February 2025}

\begin{document}
\maketitle


\textbf{7.4.3. } Decide which of the following conjectures is true, and supply a short proof. For those that are not, give a counter example.

\begin{parts}
    \part If $|f|$ is integrable on $[a, b]$, then $f$ is also integrable on this set.
    \begin{proof}
        Suppose $f: [0, 1] \rightarrow \R$ is defined as
        $$f(x) = \begin{cases}
            1, & x \in \Q \\ -1, & x \notin \Q \\
        \end{cases}$$
        Then $|f| = 1$ on [0, 1], so $|f|$ is integrable on [0, 1]. However, by theorem 7.4.2, $g(x) = \frac{1}{2}(f(x) + 1)$ should be integrable on $[0, 1]$. But $g$ is Dirichlet's function, which we know to be un-integrable from $[0, 1]$. Since in this instance, the integrability of $|f|$ and $f$ are different, we conclude this statement in false.
    \end{proof}

    \part Assume $g$ is integrable and $g(x) \geq 0$ on [a, b]. If $g(x) > 0$ for an infinite number of points $x \in [a, b]$, then $\int_a^b g > 0$.
    \begin{proof}
        Consider $g: [0, 1] \rightarrow [0, 1]$ defined as
        $$g(x) = \begin{cases}
            1, & x = \frac{1}{n}, n \in \N \\
            0, \text{otherwise} \\
        \end{cases}$$

        In HW 5 we computed $\int_0^1 g = 0$. There are clearly an infinite number of points of the form $\frac{1}{n} > 0 \in [0, 1]$, but $\int_0^1 g = 0$. We conclude this statement is false.
    \end{proof}

    \part If $g$ is continuous on $[a, b]$ and $g(x) \geq 0$ with $g(y_0) > 0$ for at least one point $y_0 \in [a, b]$, then $\int_a^b g > 0$.
    \begin{proof}
        If there exists $y_0$ s.t. $g(y_0) > 0$, then there exists $\delta > 0$ s.t. if we define $I = [a, b]\cap[y_0 - \delta, y_0 + \delta]$, then $x \in I \longrightarrow g(x) > 0$. In particular, for $\eps =\frac{g(y_0)}{2} > 0$, 
        $$g(x) \in [g(y_0) - \eps, g(y_0) + \eps] \longrightarrow g(x) > g(y_0) - \eps = \frac{g(y_0)}{2} = \eps > 0$$

        Let $c = \inf I, d = \sup I$. By theorem 7.4.1,
        \begin{align*}
            \int_a^b g &= \int_a^c g + \int_c^d g + \int_d^b g
        \end{align*}

        Since $g \geq 0$, by theorem 7.4.2 we have that $\int_a^c g, \int_d^b g \geq 0$, and also $\int_c^d g \geq \eps (d - c) > 0$

        We conclude this statement is true.
    \end{proof}
\end{parts}

\newpage
\textbf{7.4.8. }
For each $n \in \N$, let
$$h_n(x) = \begin{cases}
    \frac{1}{2^n}, & \text{if } \frac{1}{2^n} < x \leq 1 \\
    0, & \text{if } 0 \leq x \leq \frac{1}{2^n} \\
\end{cases}$$
Set $H(x) = \sum_{n = 1}^\infty h_n(x)$. Show $H$ is integrable and compute $\int_0^1 H$.
\begin{proof}
    Let $N\in \N$, let $H_N: [0, 1] \rightarrow \R$ be the $N^{th}$ partial sum of $H$. Then
    $$H_N(x) = \begin{cases}
        0, & \text{if } x\in \brac{0, \frac{1}{2^N}} \\
        \frac{2^k - 1}{2^N}, & \text{if } x\in \left (\frac{1}{2^{N - k + 1}}, \frac{1}{2^{N - k}} \right ] \\
    \end{cases}$$
    for $k \in [1, N]$. Observe that each $H_N$ is piecewise constant, thus by theorem 7.4.1 it is integrable. To explicitly compute the integral on $[0, 1]$,
    \begin{align*}
        \int_0^1 H_N &= \sum_{k = 1}^N \int_{2^{-(N - k + 1)}}^{2^{N - k}} H_N \\
        &= \sum_{k = 1}^N \paren{\frac{2^k - 1}{2^N}}\paren{\frac{1}{2^{N - k}} - \frac{1}{2^{N - k + 1}}} \\
        &= \frac{2}{3} - \frac{1}{6\cdot 4^{N - 1}} - \frac{1}{4^N} + \frac{1}{2^{N + 1}}
    \end{align*}

    Thus
    $$\int_0^1 H_N = \underset{N \rightarrow \infty}{\lim} \int_0^1 H_N = \frac{2}{3}$$
\end{proof}

\newpage
\textbf{7.5.2. }
Decide whether each statement is true or false, providing a short justification for each conclusion.

\begin{parts}
    \part If $g = h'$ for some $h$ on $[a, b]$ then $g$ is continuous on $[a, b]$.
    \sol
    This statement is false, and is apparent when we consider the function
    $$h(x) = \begin{cases}
        \sin \paren{\frac{1}{x}}, & x \neq 0 \\
        0, & x = 0 \\
    \end{cases}$$
    Here, $h$ is differentiable, but $h'$ is not continuous.

    \part If $g$ is continuous on $[a, b]$, then $g = h'$ for some $h$ on $[a, b]$.
    \sol
    Since $g$ is continuous on $[a, b]$, $g$ is integrable on $[a, b]$, so we define $h(x) = \int_a^x g$, and by the Fundamental Theorem of Calculus, we have that $h' = g$. We conclude this statement is true.

    \part If $H(x) = \int_a^x h$ is differentiable at $c\in [a, b]$, then $h$ is continuous at $c$.
    \sol
    If we consider
    $$h(x) = \begin{cases}
        0, & x \neq 0 \\
        1, & x = 0 \\
    \end{cases}$$
    Then $\int_a^x h = 0$, everywhere and thus differentiable, but $h$ is not continuous at $c = 0$. We conclude this statement is false.
    
\end{parts}
\newpage

\end{document}