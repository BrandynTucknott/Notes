\documentclass{exam}

\usepackage{amsmath,amssymb,amsfonts,amsthm,dsfont}
\usepackage{lib/extra}
\usepackage{graphicx}
\usepackage{tikz}
\usepackage{enumitem}
\usepackage{bbm}
\usepackage{pgfplots}

\title{MTH 312 HW 4}
\author{Brandyn Tucknott}
\date{4 February 2025}

\begin{document}
\maketitle

\begin{questions}
    \textbf{6.6.5. } \\
\textbf{(a) }
Generate the Taylor coefficients for the exponential function $f(x) = e^x$, then prove the corresponding Taylor series converges uniformly to $e^x$ on any interval of the form $[-R, R]$. \\
\textit{Proof. }
By Taylor's Formula, we know that the coefficients for a series of a function centered at $0$ can be computed as
$$a_n = \frac{f^{(n)}(0)}{n!} = \frac{1}{n!}$$

We represent the Taylor series of $f$ as
$$f(x) = \sum_{n = 0}^\infty \frac{x^n}{n!}$$

Let $R > 0$ and $x\in [-R, R]$. By Lagrange's Remainder Theorem, for $N\in \N$, there exists some $c \in (-R, R)$ such that $|c| < |x|$ and the error function
$$E_N(x) = f(x) - \sum_{n = 0}^N \frac{x^n}{n!}$$
satisfying
$$\abs{E_N(x)} = \abs{\frac{f^{(N + 1)}(c)}{(N + 1)!}x^{N + 1}} = \abs{\frac{e^c}{(N + 1)!}x^{N + 1}} \leq \frac{e^c}{(N + 1)!}R^{N + 1}$$
Since factorial grows faster than exponential, the error approaches 0 as $n \rightarrow \infty$, and we conclude that on $[-R, R]$, $\sum a_nx^n \rightarrow f$ uniformly. \\ \qed \\



\textbf{(b) }
Verify the formula $f'(x) = e^x$.
\sol
$$f'(x) = \sum_{n = 1}^\infty \frac{nx^{n - 1}}{n!} = \sum_{n = 1}^\infty \frac{x^{n - 1}}{(n - 1)!} = \sum_{k = 0}^\infty \frac{x^k}{k!} = e^x$$

\textbf{(c) }
Use a substitution to generate the series for $e^{-x}$, and then informally calculate $e^x\cdot e^{-x}$ by multiplying the two series and collecting powers of $x$.
\sol
$$e^x\cdot e^{-x} = \paren{1 + x + \frac{x^2}{2} + \frac{x^3}{6} + \hdots}\paren{1 - x + \frac{x^2}{2} - \frac{x^3}{6} + \hdots} =$$

$$\paren{1 - x + \frac{x^2}{2} - \frac{x^3}{6}} + \paren{1 - x + \frac{x^2}{2} - \frac{x^3}{6}}x + \paren{1 - x + \frac{x^2}{2} - \frac{x^3}{6}}\frac{x^2}{2} + \hdots =$$

$$1 + \paren{-1 + 1}x + \paren{\frac{1}{2} + \frac{1}{2} - 1}x^2 + \paren{-\frac{1}{6} + \frac{1}{2} - \frac{1}{2} + \frac{1}{6}}x^3 + \hdots = 1$$


\newpage
\textbf{6.7.3. } \\


\textbf{(a) }
Find the second degree polynomial $p(x) = q_0 + q_1x + q_2x^2$ that interpolates the three points $(-1, 1), (0, 0), (1, 1)$ on the graph of $g(x) = |x|$. Sketch $g(x)$ and $p(x)$ over $[-1, 1]$ on both axes.
\sol
The second order polynomial which passes through all given points is obviously $P(x) = x^2$. The graph can be seen below after Part (b). \\



\textbf{(b) }
Find the fourth degree polynomial that interpolates $g(x) = |x|$ at the point $x = -1, -\frac{1}{2}, 0, \frac{1}{2}, 1$. Add a sketch of this polynomial to the graph from Part (a).
\sol
We are looking for a polynomial $a(x) = a_4x^4 + a_3x^3 + a_2x^2 + a_1x + a_0$ which satisfies
$$a(-1) = 1, a(-0.5) = 0.5, a(0) = 0, a(0.5) = 0.5, a(1) = 1$$
Since $a(0) = 0$, we know that $a_0 = 0$, and we are left with 4 unknowns and 4 equations:
$$\begin{cases}
    a_4 - a_3 + a_2 - a_1 = 1 \\
    \\
    \frac{1}{16}a_4 - \frac{1}{8}a^3 + \frac{1}{4}a^2 - \frac{1}{2}a_1 = \frac{1}{2} \\
    \\
    \frac{1}{16}a_4 + \frac{1}{8}a^3 + \frac{1}{4}a^2 + \frac{1}{2}a_1 = \frac{1}{2} \\
    \\
    a_4 + a_3 + a_2 + a_1 = 1 \\
\end{cases} \longrightarrow \paren{\begin{matrix}
    -1 & 1 & -1 & 1\\
    -\frac{1}{2} & \frac{1}{4} & -\frac{1}{8} & \frac{1}{16} \\
    \frac{1}{2} & \frac{1}{4} & \frac{1}{8} & \frac{1}{16} \\
    1 & 1 & 1 & 1 \\
\end{matrix}}\paren{\begin{matrix}
    a_1 \\ a_2 \\ a_3 \\ a_4 \\
\end{matrix}} = \paren{\begin{matrix}
    1 \\ \frac{1}{2} \\ \frac{1}{2} \\ 1 \\
\end{matrix}}$$

Solving for the system above yields $a_1, a_3 = 0$, while $a_4 = -\frac{4}{3}, a_2 = \frac{7}{3}$. This gives us the final polynomial of $a(x) = -\frac{4}{3}x^4 + \frac{7}{3}x^2$.
\begin{center}
\begin{tikzpicture}
    \begin{axis}[
        axis lines = middle,
        xmin=-1.2, xmax=1.2,
        ymin=-0.2, ymax=1.2,
        xlabel={$x$},
        ylabel={$y$},
        samples=100,
        domain=-1:1,
        legend style={at={0.5, 0.5}}
    ]
        % Plot x^2
        \addplot[blue, thick] {x^2};
        \addlegendentry{$x^2$}

        % add plot a(x)
        \addplot[green, thick] {-4/3*x^4 + 7/3*x^2};
        \addlegendentry{$-\frac{4}{3}x^4 + \frac{7}{3}x^2$}

        % Plot |x|
        \addplot[red, thick] {abs(x)};
        \addlegendentry{$|x|$}
    \end{axis}
\end{tikzpicture}
\end{center}
\end{questions}

\end{document}