\documentclass{exam}

\usepackage{amsmath,amssymb,amsfonts,amsthm,dsfont}
\usepackage{lib/extra}
\usepackage{graphicx}
\usepackage{tikz}
\usepackage{enumitem}
\usepackage{bbm}
\usepackage{pgfplots}
\usepackage{fontenc}
\usepackage{float}

\pgfplotsset{compat=1.18}

\title{ECE 569 HW 2}
\author{Brandyn Tucknott}
\date{Due 28 October 2025}

\begin{document}
\maketitle

\begin{questions}
    % Question 1
    \question (Geometry of SVM) Consider the training samples $\cbrac{x_i, y_i}$. Suppose that the training samples are from two
    classes. Let us denote the samples as $\mathcal{A}\subseteq \R^n$ (class label +1) and $\mathcal{B}\subseteq\R^n$ (class label -1)
    where $\mathcal{A}\cup\mathcal{B} = \cbrac{x_1, \hdots, x_n}$. SVM aims to find a hyperplane that can separate the convex hulls
    of $\mathcal{A} = \cbrac{a_1, \hdots, a_m}$ and $\mathcal{B} = \cbrac{b_1, \hdots, b_k}$ in the $n$-dimensional space.
    \begin{parts}
        % part a
        \part [\textbf{Existence}] When does such a hyperplane exist; i.e., under what geometry of the training set does the
        SVM classifier exist? (use convex theory to support your answer)


        % part b
        \part [\textbf{Generalization}] After a hyperplane is learned, say $w^Tx = c$, which satisifies
        $$w^Ta_i \leq c \text{ for all }a_i\in\mathcal{A} \quad w^Tb_k > c \text{ for all }b_k\in \mathcal{B}.$$
    \end{parts}


    \newpage
    % Question 2
    \question


    \newpage
    % Question 3
    \question


    \newpage
    % Question 4
    \question


\end{questions}

\end{document}