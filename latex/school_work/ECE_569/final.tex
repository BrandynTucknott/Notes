\documentclass{exam}

\usepackage{amsmath,amssymb,amsfonts,amsthm,dsfont}
\usepackage{lib/extra}
\usepackage{graphicx}
\usepackage{tikz}
\usepackage{enumitem}
\usepackage{bbm}
\usepackage{pgfplots}
\usepackage{fontenc}
\usepackage{float}
\usepackage{hyperref}

\pgfplotsset{compat=1.18}

\title{Final Project: Reading Report}
\author{Brandyn Tucknott}
\date{Due 10 December 2025}

\begin{document}
\maketitle

This is a reading report on \href{https://arxiv.org/abs/0805.4471}{Exact Matrix Completion Via Convex Optimization} 
by Emmanuel J. Candes and Benjamin Recht.

\section*{Background}
As a motivator, suppose we issue a survey to a population and collect responses, forming a data matrix where the rows
represent individuals and columns denote questions. Our goal is to gather as much information as possible, but we also know that
not all individuals will answer all questions. This naturally leads us to ask: can we make a reasonable guess as to what the 
unfilled values in our data matrix should be? A more familiar version of this problem is \textit{The Netflix Problem}, where
each user has only rated a subset of all available movies, and based on those ratings a recommender system would like to guess
new movies the user might enjoy. 


\section*{Problem Formulation Rationale}
It turns out we can classify such problems as matrix completion problems. Suppose we have an $n \cross n$ matrix $M$ with
rank $r$. (Here a square matrix is used for simplicity, but neither the argument nor intuition depends on $M$ being square)
It is clear $M$ can be represented with $n^2$ numbers, so our goal is to recover $M$ using no more than $n^2$ measurements. Even
with low rank $M$, it is clear this is not generally true. For example, consider $M$ a 1-sparse matrix (matrix with only 1 element non-zero).
Then any way you sample the matrix (especially for large $n$), it is unlikely that you would stumble upon the non-zero entry 
and correctly recover the original matrix. Fortunately, while it is impossible to recover all low-rank matrices, we can still 
recover a significant amount.


Consider the singular value decomposition (SVD) of $M$ as
$$M = \sum_{k = 1}^n \sigma_k u_kv_k^*$$
where $u_k, v_k$ are the left and right singular vectors, and $\sigma_k$ are the singular values. Let 
$\mathcal{U}_r = \cbrac{u_k : 1 \leq k \leq r}$ and $\mathcal{V}_r = \cbrac{v_k : 1 \leq k \leq r}$ and define a \textit{generic} low-rank
matrix as selecting uniformly at random a family $\mathcal{U}_r, \mathcal{V}_r$ from all families of $r$ orthonormal vectors. We
make no assumptions about the depedence or independence of $\mathcal{U}_r, \mathcal{V}_r$ nor about the singular values $\sigma_k$.
This is referred to as a \textit{random orthogonal model}.


Now to address sampling, let $\Omega = \cbrac{(i,j) : 0 \leq i, j \leq n}$ where $M_{ij}$ is observed. Let $m = |\Omega|$ denote
the number of samples and assume our sampling has a random uniform distribution. Intuitively, it would be helpful if there
were only one such low-rank matrix satisfying the samples, in which case we could solve for the optimization problem
\begin{align*}
    \MIN{X} & \text{rank} (X) \\
    \subjectto & X_{ij} = M_{ij}, \text{ for all } (i,j)\in\Omega.
\end{align*}


This is unfortunately NP-hard, and all known algorithms have a doubly-exponential time complexity for an exact solution. Instead, we
consider the \textit{nuclear norm}, defined to be
$$||X||_* = \sum_{k = 1}^r \sigma_k(X)$$
where $M$ has rank $r$ and $\sigma_1(X), \hdots, \sigma_r(X)$ are the singular values of $M$ in decreasing order. Now instead of 
optimizing for rank, we choose to optimize over the nuclear norm, giving us 
\begin{align*}
    \MIN{X} & ||X||_* \\
    \subjectto & X_{ij} = M_{ij}, (i, j)\in\Omega.
\end{align*}
Intuitively the rank function counts the number of non-vanishing singular values, and the nuclear norm measures the amplitude of our singular values.
We refer to this optimization problem as OPT 1. It turns out this small change is still valid with enough samples (sufficiently large $m = |\Omega|$), but
to see why we need to introduce some theorems.

\setcounter{section}{1}
% \noqed
\begin{theorem}
    Let $M$ be an $n_1 \cross n_2$ matrix of rank $r$ sampled from the random orthogonal model, and put $n = \max (n_1, n_2)$.
    Suppose we observe $m$ entries of $M$ with locations sampled uniformly at random. Then there are numerical constants $C, c$ such that if
    $$m \geq C n^{5/4} r\log n,$$
    the minimizer to the problem OPT 1 is unique and equal to $M$ with probability at least $1 - cn^{-3}$; that is to say, the 
    semi-definite program OPT 1 recovers all the entries of $M$ with no error. In addition, if $r \leq n^{1/5}$, then recovery is exact
    with probability at least $1 - cn^{-3}$ provided that 
    $$m \geq Cn^{6/5}r\log n.$$
\end{theorem}
% \yesqed

There are a few key insights from theorem 1.1:
\begin{itemize}
    \item There is a unique low-rank matrix consistent with our observations.
    \item Such a matrix is recoverable via convex optimization (nuclear norm minimization is formally equivalent to rank minimization).
    \item Only a small number of observations is required, much less than the naive $n_1\cdot n_2$.
\end{itemize}

\begin{lemma}
    Let $U$ be a subspace of $\R^n$ with dimension $r$ and $P_U$ be the orthogonal projection onto $U$. Then the coherence of $U$
    with respect to the standard basis $e_i$ is defined to be
    $$\mu(U) \equiv \frac{n}{r}\underset{1 \leq i\leq n}{\max} ||P_Ue_i||^2.$$
\end{lemma}



\section*{Algorithm Overview}
\section*{Implementations}
\subsection*{Existing Examples}
\subsection*{New Examples}
\section*{References}
% @misc{candes2008exactmatrixcompletionconvex,
%       title={Exact Matrix Completion via Convex Optimization}, 
%       author={Emmanuel J. Candes and Benjamin Recht},
%       year={2008},
%       eprint={0805.4471},
%       archivePrefix={arXiv},
%       primaryClass={cs.IT},
%       url={https://arxiv.org/abs/0805.4471}, 
% }




\end{document}