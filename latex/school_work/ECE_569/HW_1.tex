\documentclass{exam}

\usepackage{amsmath,amssymb,amsfonts,amsthm,dsfont}
\usepackage{lib/extra}
\usepackage{graphicx}
\usepackage{tikz}
\usepackage{enumitem}
\usepackage{bbm}
\usepackage{pgfplots}
\usepackage{fontenc}
\usepackage{float}

\pgfplotsset{compat=1.18}

\title{Convex Optimization HW 1}
\author{Brandyn Tucknott}
\date{Due 17 Oct. 2025}

\begin{document}
\maketitle

\begin{questions}
    % Question 1 
    \question Assume that $C$ is an affine set. By definition, we know that for any $x_1,x_2\in C$, we have
    $$\theta x_1 + (1 - \theta)x_2 \in C, \text{ for all } \theta\in\R.$$
    Building upon the definition, show that if $x_i\in C$ for $i = 1, \hdots, n$, then we have
    $$\theta_1 x_1 + \hdots + \theta_n x_n,$$
    where $\sum_{i = i}^n \theta_i = 1.$
    \begin{proof}
        Given an affine set $C$, let $x_0\in C$ be arbitrary, and recall that $V = C - x_0$ is a subspace. Then for all $x_i\in C$
        and given $\sum_{i = 1}^n \theta_i = 1$ for $i = 1, \hdots, n$. Observe that
        \begin{align*}
            \sum_{i = 1}^n \theta_i(x_i - x_0) &\in V \\
            \sum_{i = 1}^n \theta_i(x_i - x_0) + x_0&\in C \\
            \sum_{i = 1}^n \theta_ix_i - \sum_{i = 1}^n \theta_ix_0 + x_0 &\in C \\
            \sum_{i = 1}^n \theta_ix_i - x_0\sum_{i = 1}^n \theta_i + x_0 &\in C \\
            \sum_{i = 1}^n \theta_ix_i - x_0\cdot 1 + x_0 &\in C \\
            \sum_{i = 1}^n \theta_ix_i &\in C.
        \end{align*}
    \end{proof}


    \newpage
    % Question 2
    \question Answer the following questions:
    \begin{parts}
        \part What is the distance between two parallel hyperplanes. i.e., $\cbrac{x | a^Tx = b}$ and $\cbrac{x | a^Tx = c}$?
        \begin{proof}
            Observe that
            $$\cbrac{x | a^Tx = b}, \cbrac{x | a^Tx = c} \text{ is equivalent to } \cbrac{x | a^Tx = |b - c|} \cbrac{a^Tx = 0},$$
            which geometrically gives us one hyperplane passing through the origin, and the other parallel to it. Since the shortest
            path from the origin to the hyperplane is a vector $x_0$ going directly to it (same direction as normal vector $a$), we can conclude that
            $$a^Tx_0 = |b - c|$$
            \begin{equation}\frac{a^T}{\norm{a}} x_0 = \frac{|b - c|}{\norm{a}}. \end{equation}
            Recall that 
            \begin{equation} \paren{a^T / \norm{a}}x_0 = \norm{\frac{a^T}{\norm{a}}}\norm{x_0}\cos\theta = 1\cdot \norm{x_0}\cdot 1\end{equation}
            since $(a^T / \norm{a})x_0$ is a dot product. By (1) and (2) we conclude that
            $$\norm{x_0} = \frac{|b - c|}{\norm{a}}.$$
        \end{proof}

        \part Let $a, b$ be distinct points in $\R^n$. Show that the set of all points that are closer to $a$ than $b$, i.e., 
        $\cbrac{x | \norm{x - a}_2 \leq \norm{x - b}_2}$, is a halfspace. Describe it explicitly as an inequality of the form
        $c^Tx \leq d$. Draw a picture.
        \begin{proof}
            To show the given set is a halfspace, we only need to be able to express it in form $c^Tx \leq d$, and then we will be done.
            We can algebraically manipulate the given equation:
            \begin{align*}
                \norm{x - a}_2 &\leq \norm{x - b}_2 \\
                \norm{x - a}_2^2 &\leq \norm{x - b}_2^2 \\
                x^Tx - 2x^Ta - a^Ta &\leq \x^Tx - 2x^Tb + b^Tb \\
                -2x^Ta + \norm{a}_2^2 &\leq -2x^Tb + \norm{b}_2^2 \\
                2(x^Tb - x^Ta) &\leq \norm{b}_2^2 - \norm{a_2^2} \\
                x^T(b - a)&\leq \frac{\norm{b}_2^2 - \norm{a}_2^2}{2} \\
                (b - a)^Tx&\leq \frac{\norm{b}_2^2 - \norm{a}_2^2}{2}.
            \end{align*}
            Since we have shown the given constraint is equivalent to the definition of a halfspace, we are done.
        \end{proof}
    \end{parts}


    \newpage
    % Question 3
    \question Which of the following sets are convex?
    \begin{parts}
        \part A slab $\cbrac{x\in\R^n | \alpha \leq a^Tx \leq \beta}$.
        \begin{proof}
            Since $S = \cbrac{x\in\R^n | \alpha \leq a^Tx \leq \beta}$ is the intersection of two halfspaces (which are convex), 
            we know that $S$ is convex.
        \end{proof}

        \part A rectangle $\cbrac{x\in\R^n | \alpha_i \leq x_i \leq \beta_i, i = 1, \hdots, n}$.
        \begin{proof}
            For $S = \cbrac{x\in\R^n | \alpha_i \leq x_i \leq \beta_i, i = 1, \hdots, n}$, let $y, z\in S$. Observe that
            $$\alpha_i \leq y_i \leq \beta_i,$$
            $$\alpha_i \leq z_i \leq \beta_i.$$
            Then 
            $$\theta\alpha_i \leq \theta y_i \leq \theta\beta_i,$$
            $$(1 - \theta)\alpha_i \leq (1 - \theta)z_i \leq (1 - \theta)\beta_i,$$
            which we add to see that
            \begin{align*}
                \theta\alpha_i + (1 - \theta)\alpha_i \leq \theta y_i + (1 - \theta)z_i \leq \theta z_i + (1 - \theta)z_i \\
                \alpha_i \leq \theta y_i + (1 - \theta)z_i \leq \beta_i.
            \end{align*}
            Since the convex combination is in $S$, we conclude $S$ is convex.
        \end{proof}

        \part The set of points closer to a given point than a given set:
        $$\cbrac{x | \norm{x - x_0}_2 \leq \norm{x - y}_2 \text{ for all }y\in S}$$
        where $S\subset\R^n$.
        \begin{proof}
            We can manipulate the given constraint into:
            \begin{align*}
                \norm{x - x_0}_2 &\leq \norm{x - y}_2 \\
                \norm{x - x_0}_2^2 &\leq \norm{x - y}_2^2 \\
                x^Tx - 2x^Tx_0 + x_0^Tx_0 &\leq x^Tx - 2x^Ty + y^Ty \\
                2x^T(y - x_0) &\leq \norm{y}_2^2 - \norm{x_0}_2^2 \\
                (y - x_0)^Tx &\leq \frac{\norm{y}_2^2 - \norm{x_0}_2^2}{2}.
            \end{align*}
            Since any individual $y$ yields a halfspace, to consider all $y$ we look at the intersection of the halfspaces, which
            we know to be convex.
        \end{proof}


        \part The set of points whose distance to $a$ does not exceed a fixed fraction $\theta$ of the distance to $b$,
        i.e. the set $\cbrac{x | \norm{x - a}_2 \leq \theta\norm{x - b}_2}$. You can assume $a \neq b$ and $\theta \leq 1$.
        \begin{proof}
            Again, we manipulate the constraint:
            \begin{align*}
                \norm{x - a}_2 &\leq \theta\norm{x - b}_2 \\
                \norm{x - a}_2^2 &\leq \theta^2\norm{x - b}_2^2 \\
                x^Tx - 2x^Ta + a^Ta &\leq \theta^2\paren{x^Tx - 2x^Tb + b^Tb} \\
                x^Tx - \theta^2x^Tx - 2x^Ta + 2\theta^2x^Tb &\leq \theta^2b^Tb - a^Ta \\
                (1 - \theta^2)x^Tx + 2x^T(\theta^2b - a) & \leq \norm{b}_2^2 - \norm{a}_2^2 \\
                x^Tx + \frac{2x^T (\theta^2b - a)}{1 - \theta^2} &\leq \frac{\norm{b}_2^2 - \norm{a}_2^2}{1 - \theta^2} \\
                x^Tx + \frac{2x^T*\theta^2b - a}{1 - \theta^2} + \frac{1}{(1- \theta^2)^2}x^T(\theta^2b - a)&\leq \frac{\norm{b}_2^2 - \norm{a}_2^2}{1 - \theta^2} + \frac{\frac{1}{1 - \theta^2}\norm{\theta^2b - a}_2^2}{1 - \theta^2} \\
                \norm{x + \frac{1}{1 - \theta^2}(\theta^2b - a)}_2^2 &\leq \frac{\theta^2\norm{b}_2^2 - \norm{a}_2^2 + \frac{1}{1 - \theta^2}\norm{\theta^2b - a}_2^2}{1 - \theta^2}.
            \end{align*}
            We have rewritten our constraint into the form for a ball (since RHS is constant when fixing $a, b, \theta$), and
            we already know the ball to be convex. Thus our original set is convex.
        \end{proof}
    \end{parts}


    \newpage
    % Question 4
    \question
    Show the following statements.
    \begin{parts}
        \part A polyhedron, i.e. $P = \cbrac{x | Ax \succeq b, Cx = d}$ where $A\in\R^{m x n}$ and $C\in\R^{p x n}$ is a convex set.
        \begin{proof}
            Let $x, y\in P, \theta\in [0, 1]$, and $z = \theta x + (1 - \theta)y$. Then
            \begin{align*}
                Az &= A(\theta x + (1 - \theta)y) \\
                &= \theta Ax + (1 - \theta)Ay \\
                &\succeq 0
            \end{align*}
            since $Ax \succeq 0, Ay \succeq 0, \theta, 1 - \theta \geq 0$. We also have that
            \begin{align*}
                Cz &= C(\theta x + (1 - \theta)y) \\
                &= \theta Cx + (1 - \theta)Cy \\
                &= \theta d + (1 - \theta)d \\
                &= d.
            \end{align*}
            We conclude $P$ is convex.
        \end{proof}



        \part Consider an ellipsoid $\eps = \cbrac{x | (x - x_c)^TP^{-1}(x - x_c) \leq 1}$. Assume that the eigenvalues of $P\in\R^{n x n}$
        is $\lambda_1^2, \hdots, \lambda_n^2$ in descending order. Show that the largest and smallest distances from any point on the
        boundary of the ellipsoid to $x_c$ are $\lambda_1$ and $\lambda_n$ respectively.
        \begin{proof}
            Let $P = Q\Lambda Q^T$ where $Q$ is orthogonal. Let also $y = Q^T(x - x_c)$, and note that $P^\inv = (Q\Lambda Q^T)^\inv = Q\Lambda^\inv Q^T$.
            Then we can rewrite 
            \begin{align*}
                (x - x_c)^TP^\inv (x - x_c) &= (x - x_c)^TQ\Lambda^\inv Q^T(x - x_c) \\
                &= y^T \Lambda^\inv y \\
                &= \sum_{i = 1}^n \frac{y_i^2}{\lambda_i^2} = 1.
            \end{align*}
            If we let $z_i = y_i / \lambda_i$ (for ease of representation), then we end up with $\sum_{i = 1}^n z_i^2 = 1$. Now observe that
            \begin{align*}
                \norm{y}_2^2 &= y^Ty \\
                &= (x - x_c)^TQQ^T(x - x_c) \\
                &= (x - x_c)^T(x - x_c) \\
                &= \norm{x - x_c}^2.
            \end{align*}
            Since $\norm{x - x_c} = \norm{y}$, we also have that 
            $$\norm{x - x_c}^2 = \sum_{i = 1}^n y_i^2 = \sum_{i = 1}^n \lambda_i^2z_i^2.$$
            Since $\lambda_n^2 \leq \hdots \leq \lambda_1^2$, we have that 
            \begin{align*}
                \sum_{i = 1}^n\lambda_n^2 \leq \sum_{i = 1}^n \lambda_i^2z_i^2 \leq \sum_{i = 1}^n \lambda_1^2z_i^2 \\
                \lambda_n^2 \leq \sum_{i = 1}^n \lambda_i^2z_i^2 \leq \lambda_1^2 \\
                \lambda_n^2 \leq \norm{x - x_c}^2 \leq \lambda_1^2 \\
                \lambda_n \leq \norm{x - x_c} \leq \lambda_1.
            \end{align*}
            We conclude that the largest and smallest distances to a point on the boundary of the ellipse correspond to the
            largest and smallest eigenvalues respectively.
        \end{proof}
    \end{parts}


    \newpage
    % Question 5
    \question Show the following statements.
    \begin{parts}
        \part In machine learning, we are often given training samples in the form of $(x_i, y_i)$ for $i = 1, \hdots, n$ where
        $x_i\in\R^d$ is the feature vector and $y_i\in\R$ is the label of this example. The empirical risk of Euclidean distance based
        linear regression can be expressed as follows:
        $$f(a) = \frac{1}{n}\sum_{i = 1}^n \paren{y_i - a^Tx_i}^2.$$
        Show the function $f(a)$ is convex in $a$.
        
        \part Suppose $p < 1$, $p \neq 0$. Show that the function 
        $$f(x) = \paren{\sum_{i = 1}^n x_i^p}^{1 / p}$$
        with $\dom f = \R_{++}^n$ is concave.

        \part Show that $f(X) = \tr{X^{-1}}$ is convex on $\dom f = \S_{++}^n$.
    \end{parts}



    \newpage
    % Question 6
    \question Show the conjugate of $f(X) = \tr{X^\inv} with \dom{f} = \S_{++}^n$ is given by
    $$f^*(Y) = -2\tr{-Y}^{1 / 2}, \dom{f} = -\S_{++}^n.$$
    (Hint: for unconstrained and differentiable convex problems, min / max can be found by looking for where the function
    has zero gradient.)

    % Question 7
    \question Show that the following function is convex.
    $$f(x) = x^T\paren{A(x)}^\inv x, \dom{f} = \cbrac{x | A(x) \succ 0},$$
    where $A(x) = A_0 + A_1x_1 + \hdots + A_nx_n \in\S^n$ and $A_i\in\S^n$, $i = 1, \hdots, n$. Hint: you are allowed to use
    a special form of Schur complement, described as follows: Suppose $A \succ 0$. then
    $$\paren{\begin{matrix}
        A & b \\
        b^T & c
    \end{matrix}} \succ 0 \leftrightarrow c - b^TA^\inv b \geq 0.$$
    You will need to study "epigraph" from chapter 3 of the textbook to answer this question.


\end{questions}

\end{document}