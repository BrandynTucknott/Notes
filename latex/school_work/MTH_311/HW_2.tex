\documentclass{exam}

\usepackage{amsmath,amssymb,amsfonts,amsthm,dsfont}
\usepackage{lib/extra}
\usepackage{graphicx}
\usepackage{tikz}
\usepackage{enumitem}

\title{MTH 311 Lab 2}
\author{Brandyn Tucknott}
\date{3 October 2024}

\begin{document}
\maketitle

\begin{questions}
    \question
\begin{parts}
\part
Write a formal definition of the \textit{greatest lower bound} of a set $A \subset \R$. This should be an analogue of the definition of least upper bound. The infimum of $A$ is denoted $\inf A$.
\sol
The greatest lower bound $s$ of the set $A$ is the smallest $s \in \R$ such that $s \leq a$ for all $a \in A$, and for any lower bound $b$ of $A$, $s \geq b$. \\

\part
Assume that $A$ is a nonempty set of positive real numbers.

\begin{enumerate}[label=\textit{(\roman*)}]
    \item
    Is it necessarily true that $0 < \inf A$? Explain why or why not; either give a proof or state a counterexample and explain why your example really is a counterexample.
    \sol
    Let $A = \cbrac{x \in \R : 0 < x < 1}$. Assume that $\inf A \in A$ and note that since $\inf A$ is a real number, it can be divided. If we consider $\frac{\inf A}{2}$, this is also real, an element of $A$, and in fact smaller than $\inf A$. This is a contradiction, so our assumption that $\inf A$ was an element of $A$ was wrong, and it is not necessarily true that $\inf A > 0$. \\

    
    \item
    Is it necessarily true that $0 \leq \inf A$? Explain why or why not; either give a proof or state a counterexample and explain why your example really is a counterexample.
    \sol
    First, let us write the definition of $A = \cbrac{x \in \R : x > 0}$. Then clearly any $b \leq 0 \in \R$ is a lower bound, and the greatest of these lower bounds is 0. So $\inf A = 0$, and the statement $0 \leq \inf A$ is true. Also note that if $A$ is finite or has a minimum, then we simply let $\inf A = $ minimum element of $A$, which we know exists by the well-ordering principle. We conclude it is necessarily true that $\inf A \geq 0$.
\end{enumerate}

\end{parts}

\newpage
\question
For each of the following, either give an example of what is requested (and prove that the example has the required properties), or prove that such an example is impossible.

\begin{parts}
    \part
    Two sets $A \subset \R$ and $B \subset \R$ that are bounded above, with $A \cap B = \emptyset$, sup $A$ = sup $B$, sup $A \notin A$, and sup $B \notin B$. \\
    \textit{Lemma 1. }
    If $0 < \frac{p}{q} < 1 \in \Q$, then $\frac{p + r}{q + r} > \frac{p}{q}$ for all $r > 0 \in \R$.
    \begin{proof}
        Let $\frac{p}{q} > 0 \in \Q$ and $r > 0 \in \R$. Recognize that
        $$\frac{p + r}{q + r} = \frac{p}{q + r} + \frac{r}{q + r} =$$
        $$\frac{p}{q} - \frac{rp}{q\paren{q + r}} + \frac{r}{q + r}.$$
        With this, it is sufficient to show that
        \begin{equation}
            \frac{-rp}{q\paren{q + r}} + \frac{r}{q + r} > 0.
        \end{equation}

        Since $q > 0$ and $r > 0$, we know that $q + r > 0$, and we can multiply and divide out equation (1) by $r$ and $q + r$ respectively on both sides to get
        \begin{equation}
            \frac{-p}{q} + 1 > 0.
        \end{equation}
        Since $0 < \frac{p}{q} < 1$, we know that $1 - \frac{p}{q} > 0$, we equation (2) is true. With this, we have shown that 
        equation (1) is true, and by consequence
        $$\frac{p + r}{q + r} = \frac{p}{q} - \frac{rp}{q\paren{q + r}} + \frac{r}{q + r} = \frac{p}{q} + \epsilon,$$
        for some $\epsilon > 0 \in \R$. We conclude that if $0 < \frac{p}{q} < 1$ for $\frac{p}{q} \in \Q$, then $\frac{p + r}{q + r} > \frac{p}{q}$ for all $r > 0 \in \R$. \\
    \end{proof}
    \sol
    Let $A = \cbrac{q \in \Q: q < 1}$ and $B = \cbrac{n \in \R \backslash \Q : n < 1}$. By density of $\Q$ in $\R$, for any arbitrary $q_0 < 1$, there exists $q_1 < 1$ such that $q_0 < q_1 < 1$ for $q_0, q_1 \in \Q$. It follows then, that $\sup A = 1$. Similarly, we know that for any $n \in B$, there exists $\frac{p}{q} \in \Q$ such that $n < \frac{p}{q} < 1$. It remains to be shown that the existence of a rational number $\frac{p}{q} > n$ implies the existence of an irrational $\frac{p}{q} < \frac{p + \sqrt{2}}{q + \sqrt{2}} < 1$. But this is true by \textit{Lemma 1}, so we can also conclude that $\sup B = 1$. With this we have two disjoint sets $A$ and $B$ with the same supremum which is not an element of either set. \\


    \part
    A sequence of nested unbounded closed intervals $L_1 \supset L_2 \supset L_3 \supset ...$ with $\cap_{n = 1}^\infty L_n = \emptyset$. Here, unbounded closed intervals means that each interval $L_n$ has the form $L_n = [a_n, \infty)$ for some $a_n \in \R$.
    \sol
    Choose $a_n = n$, and define $L_n = [a_n, \infty) = [n, \infty)$. We now need to show that $L_n \supset L_{n + 1}$ and $\cap_{n = 1}^\infty L_n = \emptyset$. The first case is trivial. If $L_n = [n, \infty)$, then $L_{n + 1} = [n + 1, \infty)$ is certainly a subset of $L_n$. This becomes apparent when we rewrite $[n, \infty)$ as $[n, n + 1) \cup [n + 1, \infty)$. For the case of infinite intersections, we do a proof by contradiction. Let $S = \cap_{n = 1}^\infty L_n$, and assume that $S \neq \emptyset$. Then there is some element $s \in S \rightarrow s \in L_n$ for all $n \in \N$. Now consider $L_{s + 1} = [s + 1, \infty)$. This interval does not contain $s$, a contradiction from our assumption. Thus our assumption is incorrect, and $S = \emptyset$.
\end{parts}

\newpage
\end{questions}

\end{document}