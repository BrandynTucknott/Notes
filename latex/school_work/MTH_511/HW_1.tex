\documentclass{exam}

\usepackage{amsmath,amssymb,amsfonts,amsthm,dsfont}
\usepackage{lib/extra}
\usepackage{graphicx}
\usepackage{tikz}
\usepackage{enumitem}
\usepackage{bbm}
\usepackage{pgfplots}
\usepackage{fontenc}
\usepackage{float}

\pgfplotsset{compat=1.18}

\title{MTH 511 HW 1}
\author{Brandyn Tucknott}
\date{Due 8 October 2025}

\begin{document}
\maketitle

\begin{questions}
    % Question 1 - 3.22
    \question \textbf{Exercise 3.22:} Show that $\norm{x}_\infty\leq\norm{x}_2$ for any $x\in\ell_2$, and that $\norm{x}_2 \leq \norm{x}_1$ for any $x\in\ell_1$.
        We split this proof into two parts: (a) $\norm{x}_\infty \leq \norm{x}_2$ and (b) $\norm{x}_2 \leq \norm{x}_1$.
        \begin{parts}
            \part $\norm{x}_\infty \leq \norm{x}_2$
            \begin{proof}
                Fix $k\in\N$. Then clearly
                \begin{align*}
                    |x_k|^2 &\leq \sum_{i = 1}^\infty |x_i|^2 \\
                    |x_k| &\leq \sqrt{\sum_{i = 1}^\infty} |x_i|^2 \\
                    \norm{x}_\infty = \sup_{j\in\N}{|x_j|} &\leq \sqrt{\sum_{i = 1}^\infty |x_i|^2} = \norm{x}_2
                \end{align*}
            \end{proof}

            \part $\norm{x}_2 \leq \norm{x}_1$
            \begin{proof}
                Recall that
                \begin{equation}
                    \norm{x}_1^2 &= \paren{|x_1| + |x_2| + \hdots + |x_n|}^2 = \paren{\sum_{i = 1}^\infty |x_i|}^2.
                \end{equation}
                By considering the multinomial expansion of (1), we see that
                \begin{align*}
                    \paren{\sum_{i = 1}^\infty |x_i|}^2 &\geq \sum_{i = 1}^\infty |x_i|^2 \\
                    \sum_{i = 1}^\infty |x_i| &\geq \sqrt{\sum_{i = 1}^\infty |x_i|^2} \\
                    \norm{x}_1 &\geq \norm{x}_2.
                \end{align*}
            \end{proof}
        \end{parts}



    \newpage
    % Question 2 - 3.23
    \question \textbf{Exercise 3.23:} The subset of $\ell_\infty$ consisting of all sequences that converge to 0 is denoted by $c_0$. (Note that $c_0$ is
    actually a linear subspace of $\ell_\infty$; thus $c_0$ is also a normed vector space under $\norm{\cdot}_\infty$.) Show that
    we have the following proper set inclusions: $\ell_1 \subset \ell_2 \subset c_0 \subset \ell_\infty$.
    \begin{proof}
        Recall the following:
        \begin{itemize}
        \item By definition: $c_0 \subset \ell_\infty$.
        \item By 3.22a: $\ell_2 \subset c_0$.
        \item By 3.22b: $\ell_1 \subset \ell_2$.
        \end{itemize}

        Note: recognize our proof of 3.22a relies on the fact that $(x_n)$ converges, but is independent of what the sequence
        converges to. For this reason we are able to conclude that $\ell_2 \subset c_0$ by 3.22a.

        Chaining these results together, we conclude that
        $$\ell_1 \subset \ell_2 \subset c_0 \subset \ell_\infty.$$
    \end{proof}


    \newpage
    % Question 3 - 3.25
    \question \textbf{Exercise 3.25:} Using $\norm{f}_p = \paren{\int_0^1 |f(t)|^p dt}^{1 / p}$, state and prove lemma 3.7 and theorem 3.8 (also cover $p = 1, q = \infty$ for lemma 3.7).
    \noqed
    \setcounter{section}{3}
    \setcounter{lemma}{6}
    \begin{lemma}[Holder's Inequality]\label{lemma:main}
        Let $1 < p < \infty$ and let $q$ be defined by $1 / p + 1 / q = 1$. Given $x\in \ell_p$ and $y\in\ell_q$, we have 
        $\sum_{i = 1}^\infty |x_iy_i| \leq \norm{x}_p\norm{y}_q$.
    \end{lemma}
    \yesqed
    \begin{proof}
    \end{proof}

    \noqed
    \begin{theorem}[Minkowski's Inequality]\label{thm:main}
        Let $1 < p < \infty$. If $x, y\in \ell_p$, then $x + y\in \ell_p$ and $\norm{x + y}_p \leq \norm{x}_p + \norm{y}_p$.
    \end{theorem}
    \yesqed
    \begin{proof}
    \end{proof}


    \newpage
    % Question 4 - 3.36
    \question \textbf{Exercise 3.36:} Given a metric space $(M, d)$, prove a convergent sequence is Cauchy and a Cauchy sequence is bounded.


    \newpage
    % Question 5 - 3.37
    \question \textbf{Exercise 3.37:} A Cauchy sequence with a convergent subsequence converges.


    \newpage
    % Question 6 - 3.39
    \question \textbf{Exercise 3.39:} If every subsequence of $(x_n)$ has a further subsequence that converges to $x$, then $(x_n)$ converges to $x$.


\end{questions}

\end{document}