\documentclass{exam}

\usepackage{amsmath,amssymb,amsfonts,amsthm,dsfont}
\usepackage{lib/extra}
\usepackage{graphicx}
\usepackage{tikz}
\usepackage{enumitem}
\usepackage{bbm}
\usepackage{pgfplots}
\usepackage{fontenc}
\usepackage{float}

\pgfplotsset{compat=1.18}

\title{MTH 511 HW 1}
\author{Brandyn Tucknott}
\date{Due 8 October 2025}

\begin{document}
\maketitle

\begin{questions}
    % Question 1 - 3.22
    \question \textbf{Exercise 3.22:} Show that $\norm{x}_\infty\leq\norm{x}_2$ for any $x\in\ell_2$, and that $\norm{x}_2 \leq \norm{x}_1$ for any $x\in\ell_1$.
        We split this proof into two parts: (a) $\norm{x}_\infty \leq \norm{x}_2$ and (b) $\norm{x}_2 \leq \norm{x}_1$.
        \begin{parts}
            \part $\norm{x}_\infty \leq \norm{x}_2$
            \begin{proof}
                Fix $k\in\N$. Then clearly
                \begin{align*}
                    |x_k|^2 &\leq \sum_{i = 1}^\infty |x_i|^2 \\
                    |x_k| &\leq \sqrt{\sum_{i = 1}^\infty} |x_i|^2 \\
                    \norm{x}_\infty = \sup_{j\in\N}{|x_j|} &\leq \sqrt{\sum_{i = 1}^\infty |x_i|^2} = \norm{x}_2
                \end{align*}
            \end{proof}

            \part $\norm{x}_2 \leq \norm{x}_1$
            \begin{proof}
                Recall that
                \begin{equation}
                    \norm{x}_1^2 &= \paren{|x_1| + |x_2| + \hdots + |x_n|}^2 = \paren{\sum_{i = 1}^\infty |x_i|}^2.
                \end{equation}
                By considering the multinomial expansion of (1), we see that
                \begin{align*}
                    \paren{\sum_{i = 1}^\infty |x_i|}^2 &\geq \sum_{i = 1}^\infty |x_i|^2 \\
                    \sum_{i = 1}^\infty |x_i| &\geq \sqrt{\sum_{i = 1}^\infty |x_i|^2} \\
                    \norm{x}_1 &\geq \norm{x}_2.
                \end{align*}
            \end{proof}
        \end{parts}



    \newpage
    % Question 2 - 3.23
    \question \textbf{Exercise 3.23:} The subset of $\ell_\infty$ consisting of all sequences that converge to 0 is denoted by $c_0$. (Note that $c_0$ is
    actually a linear subspace of $\ell_\infty$; thus $c_0$ is also a normed vector space under $\norm{\cdot}_\infty$.) Show that
    we have the following proper set inclusions: $\ell_1 \subset \ell_2 \subset c_0 \subset \ell_\infty$.
    \begin{proof}
        Recall the following:
        \begin{itemize}
        \item By definition: $c_0 \subset \ell_\infty$.
        \item By 3.22a: $\ell_2 \subset c_0$.
        \item By 3.22b: $\ell_1 \subset \ell_2$.
        \end{itemize}

        Note: recognize our proof of 3.22a relies on the fact that $(x_n)$ converges, but is independent of what the sequence
        converges to. For this reason we are able to conclude that $\ell_2 \subset c_0$ by 3.22a.

        Chaining these results together, we conclude that
        $$\ell_1 \subset \ell_2 \subset c_0 \subset \ell_\infty.$$
    \end{proof}


    \newpage
    % Question 3 - 3.25
    \question \textbf{Exercise 3.25:} Using $\norm{f}_p = \paren{\int_0^1 |f(t)|^p dt}^{1 / p}$, state and prove lemma 3.7 and theorem 3.8 (also cover $p = 1, q = \infty$ for lemma 3.7).
    \noqed
    \setcounter{section}{3}
    \setcounter{lemma}{6}
    \begin{lemma}[Holder's Inequality]\label{thm:main}
        Let $1 < p < \infty$ and let $q$ be defined by $1 / p + 1 / q = 1$. Given $f\in \ell_p$ and $g\in\ell_q$, we have 
        $\sum_{i = 1}^\infty |f(t)g(t)| \leq \norm{f}_p\norm{g}_q$.
    \end{lemma}
    \yesqed
    \begin{proof}
        In the case where $p = 1, q = \infty$, note if $g_{max} = \max_{t\in [0, 1]} |g(t)|$, then $|g(t)| \leq |g_{max}|$
        and we have that
        \begin{align*}
            \int_0^1 |f(t)g(t)|dt &\leq \int_0^1 |f(t)g_{max}|dt \\
            &= g_{max}\int_0^1 |f(t)|dt \\
            &= \norm{g}_\infty \norm{f}_1 = \norm{g}_q \norm{f}_p.
        \end{align*}

        If $1 < p < \infty$, then by Young's Inequality we have the following:
        \begin{align*}
            \int_0^1 \abs{\frac{f(t)g(t)}{\norm{f}_p\norm{g}_q}}dt &\leq \frac{1}{p} \int_0^1 \abs{\frac{f(t)}{\norm{f}_p}}^p dt
            + \frac{1}{q} \int_0^1 \abs{\frac{g(t)}{\norm{g}_q}}^q dt\\
            &= \frac{1}{p}\frac{1}{\norm{f}_p}^p \int_0^1 |f(t)|^p dt + 
            \frac{1}{q}\frac{1}{\norm{g}_q^q}\int_0^1 |g(t)|^q dt \\
            &= \frac{1}{p} \frac{1}{\norm{f}_p^p}\norm{f}_p^p + \frac{1}{q}\frac{1}{\norm{g}_q^q}\norm{g}_q^q \\
            &= \frac{1}{p} + \frac{1}{q} = 1.
        \end{align*}

        So clearly $\int_0^1 |f(t)g(t)|dt \leq \norm{f}_p\norm{g}_q$.
    \end{proof}



    Before we can prove Theorem 3.8, we first need to prove the analogue of Lemma 3.5 for our use.
    \setcounter{lemma}{4}
    \begin{lemma}\label{thm:main}
        Let $1 < p < \infty$ and $f, g \geq 0$. Then $(f + g)^p \leq 2^p(f^p + g^p)$. Consequently $f + g \in\ell_p$ 
        whenever $f,g\in\ell_p$.
    \end{lemma}
    \begin{proof}
        Let $1 < p < \infty$ and $f, g \geq 0$.
        \begin{align*}
            (f + g)^p &\leq (2\max(f, g))^p \\
            &= 2^p\paren{\max (f, g)}^p \\
            &= 2^p \max\paren{f^p, g^p} \\
            &\leq 2^p(f^p + g^p).
        \end{align*}

        It follows then that
        \begin{align*}
            \norm{f + g}_p = \int_0^1 |f(t) + g(t)|^p dt &\leq 2^p\paren{\int_0^1 |f(t)|^pdt + \int_0^1 |g(t)|^p dt} \\
            &= 2^p\paren{\norm{f}_p^p + \norm{g}^p} < \infty.
        \end{align*}

        Thus $f + g\in\ell_p$.
    \end{proof}
    \newpage



    \setcounter{lemma}{7}
    \noqed
    \begin{theorem}[Minkowski's Inequality]\label{thm:main}
        Let $1 < p < \infty$. If $f, g\in \ell_p$, then $f + g\in \ell_p$ and $\norm{f + g}_p \leq \norm{f}_p + \norm{g}_p$.
    \end{theorem}
    \yesqed
    \begin{proof}
        Let $1 < p < \infty$ and $f, g\in \ell_p$. By Lemma 3.5 we have that $f + g\in\ell_p$. For the inequality,
        observe that
        \begin{align*}
            \norm{f + g}_p^p &= \int_0^1 |f + g|^p dt \\
            &\leq \int_0^1 |f|^p + |g|^p t \\
            &= \int_0^1 |f|^p dt + \int_0^1 |g|^p dt \\
            &= \norm{f}_p^p + \norm{g}_p^p.
        \end{align*}
        Thus we conclude that $\norm{f + g}_p \leq \norm{f}_p + \norm{g}_p$.

    \end{proof}


    \newpage
    % Question 4 - 3.36
    \question \textbf{Exercise 3.36:} Given a metric space $(M, d)$, prove a convergent sequence is Cauchy and a Cauchy sequence is bounded.
    \begin{proof}
        First, we will show that a convergent sequence is bounded. Recall the definition for a convergent sequence:
        $$(x_n)\text{ converges to } x \leftrightarrow \text{ for all }\veps > 0, (x_n) \text{ is eventually in } B_\veps (x).$$
        Let $N\in\N$ be such that $\cbrac{x_n : n \geq N} \subset B_\veps (x)$ (we know this to be possible by the definition
        of $(x_n)$ converging). Also let $S = \cbrac{x_n : 1 \leq n < N}, m = \max (S)$. Then clearly $S \subset B_m(x)$, and we have 
        that for all $n\geq 1$,
        $$\begin{cases}
            x_n \in B_m(x), & 1 \leq n < N \\
            x_n \in B_\veps(x), &n \geq N
        \end{cases}$$
        and from it we conclude that $x_n \in B_{\max (m, \veps)}(x)$ and thus $(x_n)$ is bounded.

        Since that $\cbrac{x_n : n \geq N} \subset B_{\veps / 2}$, it follows that $\diam{B_{\veps / 2}} < \veps$, and by definition
        $(x_n)$ is Cauchy. Thus the convergent sequence $(x_n)$ is Cauchy. To show a Cauchy sequence is bounded, it follows from
        the definition, there for all $\veps > 0$ there exists $N\geq 1$ such that $\diam{x_n : n \geq N} < \veps$. Then certainly
        $\cbrac{x_n : n\geq N} \subset B_{\veps / 2}$. Since the set $\cbrac{x_n : 1 \leq n < N}$ is a finite set of finite values, it 
        is must be bounded. Since the leading terms and tail terms are bounded, the whole sequence must be bounded.

    \end{proof}


    \newpage
    % Question 5 - 3.37
    \question \textbf{Exercise 3.37:} A Cauchy sequence with a convergent subsequence converges.
    \begin{proof}
        Let $(x_n)$ be Cauchy. Suppose for all subsequence $(a_n) \subset (x_n)$, $(a_n)$ does not converge to $a$. Then by definition, 
        there exists some $\veps > 0$ where for all $N\geq 1$, there exists some $k\geq N$ such that $d(a_k, a) > \veps / 2$. Since 
        $(a_n) \subset (x_n)$, it follows then that $\diam{\cbrac{x_n : n \geq k, n}} > \veps$. This is a contradiction to our
        definition of a Cauchy sequence, so it must be that there exists a convergent subsequence.
    \end{proof}


    \newpage
    % Question 6 - 3.39
    \question \textbf{Exercise 3.39:} If every subsequence of $(x_n)$ has a further subsequence that converges to $x$, then $(x_n)$ converges to $x$.
    \begin{proof}
        We will do a proof by contrapositive. Given $(x_n)$ does not converge to $x$, it is sufficient to show then there exists a 
        subsequence where all further subsequences do not converge to $x$. Since $(x_n)$ does not converge to $x$, by definition
        we have that there exists some $\veps > 0$ where for all $N\in\N$, there is some $n_1\in\N, d(x_{n_1}, x) \geq \veps$.

        We will now inductively construct a subsequence $(a_n) \subset (x_n)$ for which $(a_n)$ has no convergent subsequences.
        Choose $a_1 = x_{n_1}$; thus $d(a_1, x) \geq \veps$. For the inductive step, assume we have chosen up to $a_k$, and wish
        to choose $a_{k + 1}$. Choose $a_{k + 1} \geq a_k + 1$, but still satisfying $d(a_{k + 1}, x)\geq \veps$. We have just constructed
        a subsequence $(a_n) \subset (x_n)$ such that for all $k\in\N, d(a_k, x)\geq \veps$. It is clear then, that any subsequence
        $(b_n) \subset (a_n)$ will have the same property, and by definition will not converge to $x$.
    \end{proof}


\end{questions}

\end{document}