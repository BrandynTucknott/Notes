\documentclass{exam}

\usepackage{amsmath,amssymb,amsfonts,amsthm,dsfont}
\usepackage{lib/extra}
\usepackage{graphicx}
\usepackage{tikz}
\usepackage{enumitem}
\usepackage{bbm}
\usepackage{pgfplots}
\usepackage{fontenc}
\usepackage{float}

\pgfplotsset{compat=1.18}

\title{Research Paper Summary: The Orienteering Problem }
\author{Brandyn Tucknott}
\date{Last Updated: 25 October 2025}

\begin{document}
\maketitle

\section{Introduction to GTSP}
Orienteering is a sport in which control points are established in an evironment, and competitors using nothing but a compass
and a map must navigate to as many control points as possible within the allotted time limit. To formalize this, given $n$ nodes
in a Euclidean plane with score $s_i \geq 0$ [$s_1 = s_n = 0$], we want to find a route through the nodes to maximize the score
beginning at 1 and ending at $n$, taking no more than TMAX time. This is referred to as the generalized traveling salesman problem (GTSP).
GTSP is NP hard, and the traveling salesman is considered a subset of this problem.

\section{Common Heuristics}
There are two general approaches to solve the GTSP, stochastic and deterministic.

\subsection{Stochastic Algorithm}
Stochastic algorithms generally rely on Monte Carlo techniques to build a large number of routes, and choosing the best one from this collection.
The thought is this: if $A_j$ is a measure of "desirability" for nodes $j$ currently not on the route, then we say
$$A_j = \frac{s_j}{t(\text{last}, j)},$$
where $s_j$ is the score associated with node $j$ and $t(\text{last}, j)$ is the travel time from the last node to $j$. After choosing
at most four values for $A_j$, we normalize them, and a random number from 0 to 1 is generated to determine which $j$ node is included. This is 
repeated until no additional nodes can be included in the route.

\subection{Deterministic ALgorithm}
This approach creates routes using a variant of Wren-Holliday vehicle routing procedure. The environment is divided into sectors 
using concentric circles, and routes are built up from within sectors to save travel time.

\section{Center of Gravity Heuristic}
This new proposed heuristic has three core steps:
\begin{enumerate}
    \item Route Construction Step
    \item Route Improvement Step
    \item Center of Gravity Step
\end{enumerate}

\subection{Route Construction}
The goal of this step is to find a route starting at 1 and ending at $n$ which has a high score while remaining within the time contraints.
We form a route using an insertion heuristic, with the best condidates being those with high score that don't add too much to the duration.

\subsection{Route Improvement}
This step improves upon the route generated in the previous step using an interchange procedure such as 2-OPT to find a shorter route passing through the same set of nodes.
This is followed by a cheap insertion, inserting as many nodes as possible without going over TMAX. Call this new route $L$.

\subection{Center of Gravity}
Suppose that node $i$ has coordinates $(x_i, y_i)$. We can then calculate the center of gravity of $L$ as $g = (\overline{x}, \overline{y})$, where
\begin{align*}
    \overline{x} &= \frac{\sum_{i\in L} s_ix_i}{\sum_{i\in L} s_i} \\
    \overline{y} &= \frac{\sum_{i\in L} s_iy_i}{\sum_{i\in L} s_i}
\end{align*}

Now for $i = 1, \hdots, n$, let $a_i = t(i, g)$. We form a route from 1 to $n$ in the following way:
\begin{parts}
    \part Calculate the ratio $s_i / a_i$ for all $i$ (reward / cost).
    \part Add nodes to the route in decreasing order of this ratio (add nodes with best reward / cost ratio).
    \part Use the route improvement step to make adjustments to the resulting route.
\end{parts}

At the end of all this, we end up with a route $L_1'$. Finding the center of gravity for this route repeats (a) through (c), resulting in another route $L_2'$.
repeat this cycle until routes $L_p'$ and $L_q'$ are identical for $q > p$. Finally, we select the route with the highest score from the collection
of routes $\cbrac{L_1', \L_2', \hdots, L_q'}$. This requires $q \leq 10$, but in practice it has been observed to be $q \leq 5$.


\end{document}