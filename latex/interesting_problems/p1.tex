\documentclass{exam}

\usepackage{amsmath,amssymb,amsfonts,amsthm,dsfont}
\usepackage{lib/extra}
\usepackage{graphicx}
\usepackage{tikz}
\usepackage{enumitem}
\usepackage{bbm}
\usepackage{pgfplots}
\usepackage{fontenc}
\usepackage{float}

\pgfplotsset{compat=1.18}

\title{Problem 1: Dot Product of Random Vectors}
\author{Brandyn Tucknott}
\date{Last Updated: 7 November 2025}

\begin{document}
\maketitle

\begin{parts}
    \part If $v, u\in\R^n$ for $n < \infty$ with
    $$v_i = \begin{cases}
        1, & \text{ with probability 1/2} \\
        0, & \text{ with probability 1/2}
    \end{cases}$$
    $$u_i = \begin{cases}
        1, & \text{ with probability 1/4} \\
        0, & \text{ with probability 3/4}
    \end{cases},$$
    what is the expected value of $v\cdot u$ (with proof)?
    \begin{proof}
        We assume that the $u_i, v_i$ are independent of each other since no reason has been given to lead us to suspect otherwise.
        Then by linearity of expectation and independence, we have that
        \begin{align*}
            \E{u \cdot v} &= \sum_{i = 1}^n \E{u_i \cdot v_i} \\
            &= \sum_{i = 1}^n \p{u_i = 1, v_i = 1} \\
            &= \sum_{i = 1}^n \p{u_i = 1}\p{v_i = 1} \\
            &= \sum{i = 1}^n \frac{1}{4}\cdot \frac{1}{2} \\
            &= \frac{n}{8}.
        \end{align*}
    \end{proof}



    \part What about for a more generalized case?
    $$v_i = \begin{cases}
        1, & \text{ with probability } p \\
        0, & \text{ with probability } 1 - p
    \end{cases}$$
    $$u_i = \begin{cases}
        1, & \text{ with probability } q \\
        0, & \text{ with probability } 1 - q
    \end{cases},$$
    what is the expected value of $v\cdot u$ (with proof)?
    \begin{proof}
        Again assuming independence, we use linearity to find that
        \begin{align*}
            \E{u\cdot v} &= \sum_{i = 1}^n \E{u_i\cdot v_i} \\
            &= \sum_{i = 1}^n \p{u_i = 1, v_i = 1} \\
            &= \sum_{i = 1}^n \p{u_i = 1}\p{v_i = 1} \\
            &= \sum_{i = 1}^n pq \\
            &= npq.
        \end{align*}
    \end{proof}

\end{parts}


\end{document}