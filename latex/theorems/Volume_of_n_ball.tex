\documentclass{exam}

\usepackage{amsmath,amssymb,amsfonts,amsthm,dsfont}
\usepackage{lib/extra}
\usepackage{graphicx}
\usepackage{tikz}
\usepackage{enumitem}
\usepackage{bbm}
\usepackage{pgfplots}
\usepackage{fontenc}
\usepackage{float}

\pgfplotsset{compat=1.18}

\title{Volume of a N-Dimensional Ball}
\author{Brandyn Tucknott}
\date{Last Updated: 14 April 2025}

\begin{document}
\maketitle

\textbf{Theorem (Volume of an N-Ball)} The volume of an 
$n$-dimensional ball with radius $r$ can be expressed as follows:
$$\text{($B^n(r)$)} = \frac{\pi^{n / 2}}{\Gamma \paren{\frac{n}{2}}\cdot \frac{n}{2}}r^n$$

\begin{proof}
    Let $\omega_n$ be the surface characteristic for a $n - 1$ dimensional 
    sphere. Then we can express the volume of a $n$ dimensional ball with radius $r$
    as
    \begin{align*}
        \text{Vol}\paren{B^n(r)} &= \int_{B^n(r)}dr \\
        &= \int_0^r \omega_n r^{n - 1}dr \\
        &= \omega_n \frac{r^n}{n}
    \end{align*}

    Our new goal is to find an expression for the surface characteristic, which will
    complete our expression for the volume. Consider the multidimensional Gaussian,
    and observe that
    \begin{align*}
        \npint e^{-\norm{\mathbf{x}}_2^2} d\mathbf{x} &= \npint e^{-x_1^2 - \hdots - x_n^2}dx_1\hdots dx_n \\
        &= \underbrace{\npint e^{-x_1^2} dx_1 \hdots \npint e^{-x_n^2}dx_n}_{n \text{ times}} \\
        &= \paren{\sqrt{\pi}}^n = \pi^{n / 2}
    \end{align*}

    However, since it is a radial function, we can also evaluate the 
    multidimensional Gaussian as
    \begin{align*}
        \npint e^{-\norm{\mathbf{x}}_2^2} d\mathbf{x} &= \omega_n \zpint e^{-r^2}r^{n - 1} dr \\
        &= \frac{\omega_n}{2} \zpint e^{-r^2} r^{n - 2}\cdot 2r dr \text{ set $t = r^2 \rightarrow dt = 2r$} \\
        &= \frac{\omega_n}{2} \zpint e^{-t} t^{(n - 2) / 2} dt \\
        &= \frac{\omega_n}{2} \zpint e^{-t}t^{n / 2 - 1} dt \\
        &= \frac{\omega_n}{2} \Gamma\paren{n / 2}
    \end{align*}

    Setting the two results equal to each other, we can arive at a value for $\omega_n$.
    \begin{align*}
        \frac{\omega_n}{2} \Gamma (n / 2) &= \pi^{n / 2} \\
        \omega_n &= \frac{2\pi^{n / 2}}{\Gamma (n / 2)} \\
    \end{align*}

    Finally, we substitute this into our original volume expression, and conclude that
    \begin{align*}
        \text{Vol}(B^n(r)) &= \omega_n\frac{r^n}{n} \\
        &= \frac{2\pi^{n / 2}}{\Gamma \paren{\frac{n}{2}}} \frac{r^n}{n} \\
        &= \frac{\pi^{n / 2}}{\Gamma \paren{\frac{n}{2}} \frac{n}{2}} r^n
    \end{align*}
    
\end{proof}
\end{document}