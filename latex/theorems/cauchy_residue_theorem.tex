\documentclass{exam}

\usepackage{amsmath,amssymb,amsfonts,amsthm,dsfont}
\usepackage{lib/extra}
\usepackage{graphicx}
\usepackage{tikz}
\usepackage{enumitem}
\usepackage{bbm}
\usepackage{pgfplots}
\usepackage{fontenc}
\usepackage{float}

\pgfplotsset{compat=1.18}

\title{Cauchy Residue Theorem}
\author{Brandyn Tucknott}
\date{Last Updated: 7 November 2025}

\begin{document}
\maketitle

\setcounter{section}{1} % no real purpose, just to make it look better
First, we introduce the concept of a residue.
\begin{theorem}
    If $f$ has a pole of order $n$ at $z_0$, then
    $$\res_{z_0}(f) = \lim_{z\to z_0} \frac{1}{(n - 1)!}\frac{d^{n - 1}}{dz^{n - 1}}f(z)(z - z_0)^n.$$
\end{theorem}
Next, the theorem:
\begin{theorem}
    Suppose that $f$ is holomorphic in an open set containing a circle $C$ and its interior, except for a pole at $z_0$ inside $C$. Then
    $$\int_C f(z) dz = 2\pi i \res_{z_0} f.$$
\end{theorem}




\end{document}